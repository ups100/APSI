\documentclass[a4paper,onecolumn,10pt]{article}

\usepackage{polski}
\usepackage[utf8]{inputenc}
\usepackage{tgtermes}
\usepackage{tgcursor}
\usepackage[QX]{fontenc}
\usepackage[T1]{fontenc}
\usepackage{longtable}
\usepackage{fancyhdr}
\usepackage{fancyvrb}
\usepackage{epsfig}
\usepackage{color}
\usepackage{listings} 
\usepackage{multicol}
\usepackage{capt-of}
\usepackage{url}
\usepackage{epstopdf}
\usepackage{tabularx}
\usepackage{enumitem}

% % % % % % % % % zwiększenie możliwości zagnieżdżenia wyliczeń : http://tex.stackexchange.com/questions/41408/a-five-level-deep-list
\setlistdepth{7}

\newlist{myEnumerate}{enumerate}{7}
\setlist[myEnumerate,1]{label*=\arabic*.}
\setlist[myEnumerate,2]{label*=\arabic*.}
\setlist[myEnumerate,3]{label*=\arabic*.}
\setlist[myEnumerate,4]{label*=\arabic*.}
\setlist[myEnumerate,5]{label*=\arabic*.}
\setlist[myEnumerate,6]{label*=\arabic*.}
\setlist[myEnumerate,7]{label*=\arabic*.}
% % % % % % % % % % % % % % % % % % % % % % % % % % % % % % % % %
%% Two counters to keep track of categories and 
%% subcategories

% % % % % % % % % % % % % % % % % % table
\usepackage{etoolbox}
\newenvironment{tabularlist}[1]{%
	\preto{\tabular}{\renewcommand{\item}{\label{#1:\thetblenum} &} \setcounter{tblenum}{0}}
	\setlength{\extrarowheight}{.5\baselineskip}
	\begin{tabular}}
	{\end{tabular}}
\newcounter{tblenum}
\newlength{\tblenumlabelspace}
\setlength{\tblenumlabelspace}{.75em}
\newcolumntype{T}[1]{@{}>{\refstepcounter{tblenum}\thetblenum.}
	r@{\hspace{\tblenumlabelspace}}p{#1\linewidth}}
% % % % % % % % % % % % % % % % % table

\usepackage{listings}
\usepackage{color}
\usepackage{lastpage}

\setlength{\headheight}{12.02503pt}

\usepackage{hyperref}
\usepackage{fancyhdr}

\usepackage{longtable}
\usepackage{float}

\begin{document}
\pagestyle{fancy}
\lhead{Moduł rejestru dostępnych zasobów}
\chead{}
\rhead{APSI}
\lfoot{Dokumentacja projektowa}
\cfoot{wersja 0.1}
\rfoot{\thepage\ / \pageref{LastPage}}
\renewcommand{\headrulewidth}{0.4pt}
\renewcommand{\footrulewidth}{0.4pt}

\title{
\begin{huge}
	\textit{\textbf{Moduł rejestru dostępnych zasobów}}
\end{huge}
\begin{small}
	\\ Dokumentacja projektowa \vspace{135mm}
\end{small}}

\author{
\begin{tabular}{l l}	
		Zespół \textbf{A07/14Z}: & \\ \\ 
		Dziurdziak Mateusz &  Niedźwiedź Andrzej  \\ 
 	Gadawski Łukasz &  	Opasiak Krzysztof  \\ 
 	Marcinkowski Łukasz & \\ \\ \\  
\end{tabular}
}

\maketitle
\thispagestyle{empty}
\tableofcontents

%\chapter{pierwsz}
\section{Wstęp}

% Wszelkie podobieństwo do realnych firm a tym bardziej fikcyjnych jest przypadkowe :D:D:D
Duża firma informatyczna {\bf Nabiano} ze względu na swój dynamiczny
rozwój zleciła opracowanie zintegrowanego systemu wspomagającego
zarządzanie pracownikami oraz wszelkimi zasobami wykorzystywanymi w
firmie.

Ponieważ powstający system jest bardzo rozbudowany, główny wykonawca
postanowił zastosować w nim architekturę modularną. W skład systemu
wchodzą następujące moduły:

\begin{itemize}
\item[--] moduł repozytorium dokumentów wraz z mechanizmem obiegu dokumentów,
\item[--] moduł rejestru pracowników i wykonywanych prac,
\item[--] moduł rejestru dostępnych zasobów,
\item[--] moduł alokacji zasobów i planowania obsady projektów,
\item[--] moduł repozytorium wymagań dla projektów,
\item[--] moduł repozytorium testów,
\item[--] moduł repozytorium problemów technicznych.
\end{itemize}

Projekt oraz implementacja poszczególnych modułów została przekazana
podwykonawcom. Jednym z nich jest firma {\bf ELKA-Infor}. Jest ona
odpowiedzialna za przygotowanie projektu modułu rejestru dostępnych
zasobów oraz jego implementację.

Niniejszy dokument został przygotowany przez podwykonawce (ELKA-Infor)
jako dokumentacja projektowa do tworzonego modułu rejestru dostępnych
zasobów. Dokument ten ma na celu ułatwić współprace z głównym
wykonawcą, a także pozwolić na specyfikację wymagań wobec innych
modułów wchodzących w skład systemu.

\subsection{Opis działalności}

Firma Nabiano działa na polskim rynku od dziesięciu lat. Została ona
założona przez jej obecnego prezesa Pana Siergija Wybora. We wczesnej
fazie rozwoju firma zajmowała się sprzedażą sprzętu komputerowego
oraz oprogramowania w kilku punktach w Białymstoku. W roku 2007 firma
rozszerzyła znacząco zakres swojej działalności i ukierunkowała się na
klienta biznesowego poprzez wprowadzenie do swojej oferty sprzętu
serwerowego oraz oprogramowania CMS firmy Atlasin.

Znaczący wzrost liczby zatrudnionych pracowników miał miejsce w 2009
roku, kiedy to firma postanowiła zainwestować duże środku i przy
wsparciu funduszy unijnych otworzyła dział rozwojowy w Łodzi, którego celem
było stworzenie własnego systemu CMS. Pierwsza wersja systemu powstała
już w 2010 roku i zakupiło ją kilka firm oraz jedna z największych w
Polsce korporacji.

Kolejnym krokiem milowym w rozwoju firmy było podpisanie w 2012 roku
kontraktu z rządem Korei Północnej na dostarczenie elektronicznego
systemu wyborczego na wybory parlamentarne w 2014. Projekt ten
stanowił dla firmy ogromne wyróżnienie oraz wyzwanie. Aby podołać temu
trudnemu zleceniu zwiększono zatrudnienie do ponad 1000 pracowników
oraz otwarto nowe oddziały w Warszawie oraz Pjongjang. W styczniu 2014
roku pomyślnie wdrożono wspomniany wcześniej system wyborczy. W marcu
tego samego roku odbyły się przy użyciu tego systemu wybory
parlamentarne. System działał bez żadnych zakłóceń oraz awarii. Po
wygranych przez {\it Kim Dzong Una} wyborach parlamentarnych firma
otrzymała list gratulacyjny oraz obietnicę polecenia tego systemu
wyborczego przywódcą innych państw. Już po pół roku od udanego
wdrożenia systemu wyborczego w Korei do firmy zaczęły napływać
zamówienia z całego świata, między innymi z Białorusi, Kuby oraz
Rosji. Swoje zainteresowanie wyraziły również inne państwa
europejskie, które mają problemy ze swoimi systemami wyborczymi.

W chwili obecnej firma Nabiano ponownie zwiększa swoje zatrudnienie i
rozpoczyna realizację kolejnych projektów. Na początku 2015 roku
planowane jest otwarcie oddziału firmy w Moskwie, a w 2016 roku w
Mińsku. Ze względu na ten dynamiczny rozwój firma postanowiła wdrożyć
system wspomagający zarządzania pracownikami, projektami oraz
wszystkimi środkami firmy.

\subsection{Przeznaczenie systemu}

W chwili obecnej w posiadaniu firmy znajduje się:

\begin{itemize}
\item[--] około 50 pojazdów
\item[--] około 800 telefonów
\item[--] około 1500 komputerów stacjonarnych
\item[--] około 2500 monitorów komputerowych
\item[--] około 2000 laptopów
\item[--] około 100 serwerów
\item[--] około 200 sztuk sprzętu serwerowego
\item[--] około 10000 sztuk innego drobnego sprzętu komputerowego
\item[--] około 300 nośników z oprogramowaniem
\item[--] około 200 licencji na użytkowanie oprogramowania (jedna licencja na wiele stanowisk)
\end{itemize}

Zarządzanie tymi zasobami w chwili obecnej wymaga znacznego wkładu
pracowników. Firma zatrudnia obecne około 40 pracowników, którzy
zajmują się zarządzaniem jej środkami. Wśród tych pracowników znajduje
się również grupa odpowiedzialna za komunikację z zewnętrznymi firmami
serwisującymi posiadane urządzenia. Pracownicy Ci rozlokowani są w
różnych oddziałach firmy więc ich podstawowym problemem jest brak
możliwości bezpośredniej komunikacji i współpracy.

Bieżący system zarządzania środkami oparty jest o setki arkuszy
kalkulacyjnych trzymanych na wewnętrznym serwisie firmowym. Taki
sposób pracy bardzo utrudnia śledzenie zarówno bieżącego stanu
sprzętowego (utrudniona agregacja) oraz powiązanie pomiędzy problemami
zgłaszanymi przez pracowników, a zleceniami serwisowania
sprzętu. Istotnym problemem jest również śledzenie zmian obecnych
użytkowników sprzętu, a także historii jego przekazywania pomiędzy
pracownikami.

Ze względu na wspomniane problemu firma postanowiła przy wdrożeniu
nowego systemu wspierającego jej funkcjonowanie utworzyć również moduł
zarządzania zasobami opisany w tym dokumencie. Moduł ten ma rozwiązać
wszystkie problemy obecne w obecnym systemie. W celu umożliwienia
prawidłowego funkcjonowania całego systemu moduł ten musi
współpracować z modułem pracowników w zakresie własności oraz
bieżącego użytkowania oraz z repozytorium problemów w celu śledzenia
problemów zgłaszanych przez użytkowników oraz śledzenia realizacji
zleceń serwisowych.


\section{Analiza wymagań}

\subsection{Wymagania funkcjonalne}
%					\\ \rule{\linewidth}{.1pt}
Lista wymagań funkcjonalnych wraz z priorytetami gdzie 1 oznacza najwyższy priorytet, natomiast 3 najniższy. \\

\hfill Priorytet
\begin{myEnumerate}
\item\label{f1} \textbf{Zarządzanie zasobami.}
\hfill 1
	\begin{myEnumerate}
	\item\label{f2} Przeszukiwanie zasobów.
	\hfill 1
	\item\label{f3} Katalogowanie zasobów.
	\hfill 1
	\begin{myEnumerate}

	\item\label{f4} Obsługa sprzętu komputerowego.
	\hfill 1
	\begin{myEnumerate}
		\item\label{f5} Dodanie zasobu.
		\hfill 1
		\begin{myEnumerate}
		\item\label{f6} Dodanie podstawowych informacji o zasobie.
		 \hfill 1
		\item\label{f7} Dodanie informacji o elementach wchodzących w skład \\  zasobu.
		\hfill 1
		\item\label{f8} Zapisanie konfiguracji sprzętu komputerowego.
		\hfill 2
		\end{myEnumerate}
		\item\label{f9} Edycja informacji o zasobie.
		\hfill 2
		\item\label{f10} Usunięcie zasobu z systemu.
		\hfill 3
	\end{myEnumerate}

	\item\label{f11} Obsługa oprogramowania.
	\hfill 1

	\begin{myEnumerate}
		\item\label{f12} Dodanie zasobu.
		\hfill 1
		\begin{myEnumerate}
			\item\label{f13} Dodanie podstawowych informacji o zasobie.
			\hfill 1
			\item\label{f14} Dodanie informacji o instalacji oprogramowania.
			\hfill 2
		\end{myEnumerate}
		\item\label{f15} Edycja informacji o zasobie.
		\hfill 2
		\item\label{f16} Usunięcie zasobu z systemu.
		\hfill 3
	\end{myEnumerate}

	\item\label{f17} Obsługa innego sprzętu, urządzeń i wyposażenia.
	\hfill 1
	\begin{myEnumerate}
		\item\label{f18} Dodanie zasobu.
		\hfill 1
		\begin{myEnumerate}
			\item\label{f19} Dodanie podstawowych informacji o zasobie.
			\hfill 1
		\end{myEnumerate}
		\item\label{f20} Edycja informacji o zasobie.
		\hfill 2
		\item\label{f21} Usunięcie zasobu z systemu.
		\hfill 3
	\end{myEnumerate}

	\item\label{f22} Obsługa czasopism oraz literatury.
	\hfill 1
	\begin{myEnumerate}
		\item\label{f23} Dodanie zasobu.
		\hfill 1
		\begin{myEnumerate}
			\item\label{f24} Dodanie podstawowych informacji o zasobie.
			\hfill 1
		\end{myEnumerate}
		\item\label{f25} Edycja informacji o zasobie.
		\hfill 2
		\item\label{f26} Usunięcie zasobu z systemu.
		\hfill 3
	\end{myEnumerate}

	\end{myEnumerate}

	\item\label{f27} Współpraca z modułem rejestru pracowników.
	\hfill 1
	\begin{myEnumerate}
	\item\label{f28} Zapisanie informacji o osobie odpowiedzialnej za dany zasób.
	\hfill 1
	\item\label{f29} Zapisanie informacji o użytkowniku konkretnego
	 zasobu.
	 \hfill 1
	\end{myEnumerate}

	\item\label{f30} Obliczanie statystyk.
	\hfill 3
	\begin{myEnumerate}
	\item\label{f31} Prezentacja zakupów konkretnych zasobów w poszczególnych \\ latach.
	\hfill 3
	\item\label{f32} Prezentacja ilości zasobów w poszczególnych działach.
	\hfill 3
	\item\label{f33} Prezentacja ilość zasobów w poszczególnych placówkach.
	\hfill 3
	\end{myEnumerate}
	\end{myEnumerate}


\item\label{f34} \textbf{Zarządzanie serwisem sprzętu oraz oprogramowania.}
	\hfill 1
	\begin{myEnumerate}
		\item\label{f35} Rejestracja informacji dotyczącej miejsca zakupu.
			\hfill 1
		\item\label{f36} Historia napraw sprzętu.
		\hfill 1
			\begin{myEnumerate}
			\item\label{f37} Rejestracja informacji o wewnętrznej naprawie sprzętu.
			\hfill 1
			\item\label{f38} Rejestracja naprawy sprzętu przez serwis zewnętrzny.
			\hfill 1
			\end{myEnumerate}
		\item\label{f39} Historia obsługi oprogramowania.
		\hfill 1
		\begin{myEnumerate}
		\item\label{f40} Rejestracja aktualizacji oprogramowania.
			\hfill 1
		\item\label{f41} Rejestracja "naprawy" oprogramowania.
					\hfill 1
		\end{myEnumerate}
		\item\label{f42} Współpraca z repozytorium problemów.
					\hfill 1
		\begin{myEnumerate}
		\item\label{f43} Rejestracja operacji serwisowej.
					\hfill 1
		\end{myEnumerate}
	\end{myEnumerate}
\end{myEnumerate}

% ostatnia etykieta labelki: f43

\subsection{Wymagania niefunkcjonalne}


\begin{myEnumerate}
	\item \textbf{Bezpieczeństwo}
	\begin{myEnumerate}
		\item Przechowywanie danych użytkowników zgodnie z wymogami ustawy o ochronie danych osobowych.
		\item Szyfrowanie danych wrażliwych dotyczących użytkowników zasobów.
		\item Komunikacja z systemem z wykorzystaniem SSL.
		\item Hasła użytkowników przetrzymywane w formacie bezpiecznym - w postaci wyniku funkcji skrótu hasła połączonego z losowo wygenerowaną solą.
		\item Integracja z firmowym serwerem LDAP.
	\end{myEnumerate}
	\item \textbf{Dostępność}
	\begin{myEnumerate}
		\item Dostępność systemu na poziomie 99\%.
	\end{myEnumerate}
	\item \textbf{Niezawodnosć}
	\begin{myEnumerate}
		\item Możliwość wykonywania kopii zapasowych.
		\item Możliwość zautomatyzowania procesu wykonywania kopii zapasowych.
		\item Maksymalny czas restartu systemu 2h.
		\item Możliwość odtworzenia systemu z kopii zapasowej w czasie poniżej 6h.
		\item Transakcyjny charakter operacji.
	\end{myEnumerate}
	\item \textbf{Użyteczność}
	\begin{myEnumerate}
		\item Intuicyjny interfejs graficzny.
		\item Wbudowana pomoc kontekstowa dla użytkowników systemu.
		\item Przygotowanie materiałów szkoleniowych.
		\item System dostępny w wersjach językowych: polskiej, koreańskiej.
	\end{myEnumerate}
	\item \textbf{Elastyczność}
	\begin{myEnumerate}
		\item Architektura systemu zapewnia możliwość dodawania nowych funkcji.
		\item Architektura systemu zezwala na zintegrowanie go z innymi systemami.
		\item Możliwość skalowania systemu.
	\end{myEnumerate}
	\item \textbf{Wydajność}
	\begin{myEnumerate}
		\item System przystosowany jest do równoczesnej pracy 100 pracowników.
		\item Czas średniej odpowiedzi systemu na zapytanie powinien być krótszy niż 300 ms.
	\end{myEnumerate}
\end{myEnumerate}

\subsection{Specyfikacja przypadków użycia - poziom ogólny}

\begin{table}[!ht]
	\begin{center}
		{\Large\bf Wyszukiwanie zasobów} \\
	\end{center}
\begin{tabular}{| c | c | p{.20\textwidth} | p{.40\textwidth} |}
	\hline \textbf{ID} & \textbf{Wymaganie} & \textbf{Nazwa} & \textbf{Opis} \\
	\hline UC1 & 1.1. & Wyszukiwanie zasobów  & Użytkownik wyszukuje zasoby podając kryteria wyszukiwania \\
	\hline
\end{tabular}
\end{table}

\begin{table}[!ht]
	\begin{center}
		{\Large\bf Obsługa sprzętu komputerowego} \\
	\end{center}
\begin{tabular}{| c | c | p{.20\textwidth} | p{.40\textwidth} |}
	\hline \textbf{ID} & \textbf{Wymaganie} & \textbf{Nazwa} & \textbf{Opis} \\
	\hline UC2.1 & 1.2.1.1. & Dodanie nowego sprzętu komputerowego do systemu & Użytkownik dodaje nowy sprzęt komputerowy do systemu katalogowania, podając podstawowe informacje o sprzęcie, skład elementów wchodzących w jego skład oraz zapisuje konfiguracje sprzętu \\
	\hline UC2.2 & 1.2.2 & Edycja informacji o sprzęcie komputerowym & Użytkownik edytuje informacje dotyczące sprzętu komputerowego, który już znajduje się w systemie - może zmienić podstawowe informacje o sprzęcie, skład elementów wchodzących w jego skład oraz zmienić konfiguracje sprzętu \\
	\hline UC2.3 & 1.2.3. & Usunięcie sprzętu komputerowego z systemu & Użytkownik usuwa sprzęt komputerowy z systemu \\
	\hline
\end{tabular}
\end{table}

\begin{table}[!ht]
	\begin{center}
		{\Large\bf Obsługa oprogramowania} \\
	\end{center}
\begin{tabular}{| c | c | p{.20\textwidth} | p{.40\textwidth} |}
	\hline \textbf{ID} & \textbf{Wymaganie} & \textbf{Nazwa} & \textbf{Opis} \\
	\hline UC3.1 & 1.3.1. & Dodanie nowego oprogramowania do systemu & Użytkownik dodaje nowego oprogramowanie do systemu katalogowania, podając podstawowe informacje o oprogramowaniu oraz zapisując informację na temat przeprowadzonych instalacji \\
	\hline UC3.2 & 1.3.2. & Edycja informacji o oprogramowaniu & Użytkownik edytuje informacje dotyczące oprogramowania znajdującego się w systemie - może zmienić podstawowe informacje o oprogramowaniu oraz edytować informację na temat przeprowadzonych instalacji \\
	\hline UC3.3 & 1.3.3. & Usunięcie oprogramowania z systemu & Użytkownik usuwa informacje dotyczące oprogramowania z systemu  \\
	\hline
\end{tabular}
\end{table}

\begin{table}[!ht]
	\begin{center}
		{\Large\bf Obsługa innego sprzętu, urządzeń i wyposażenia} \\
	\end{center}
\begin{tabular}{| c | c | p{.20\textwidth} | p{.40\textwidth} |}
	\hline \textbf{ID} & \textbf{Wymaganie} & \textbf{Nazwa} & \textbf{Opis} \\
	\hline UC4.1 & 1.4.1. & Dodanie innego sprzętu, urządzeń lub wyposażenia do systemu & Użytkownik dodaje nowy sprzęt, urządzenie lub wyposażenie do systemu \\
	\hline UC4.2 & 1.4.2. & Edycja informacji o innym sprzęcie, urzędzeniu lub wyposażeniu & Użytkownik edytuje informacje dotyczące innego sprzętu, urządzenia lub wyposażenia znajdującego się w systemie \\
	\hline UC4.3 & 1.4.3. & Usunięcie innego sprzętu, urządzenia lub wyposażenia z systemu & Użytkownik usuwa informacje dotyczące innego sprzętu, urządzenia lub wyposażenia z systemu \\
	\hline
\end{tabular}
\end{table}

\begin{table}[!ht]
	\begin{center}
		{\Large\bf Obsługa czasopism oraz literatury} \\
	\end{center}
\begin{tabular}{| c | c | p{.20\textwidth} | p{.40\textwidth} |}
	\hline \textbf{ID} & \textbf{Wymaganie} & \textbf{Nazwa} & \textbf{Opis} \\
	\hline UC5.1 & 1.5.1. & Dodanie nowego czasopisma lub zasobu literaturowego do systemu & Użytkownik dodaje nowego czasopismo lub zasób literaturowy do systemu katalogowania \\
	\hline UC5.2 & 1.5.2. & Edycja informacji o czasopiśmie lub zasobie literaturowym & Użytkownik edytuje informacje dotyczące czasopisma lub zasobu literaturowego znajdującego się w systemie \\
	\hline UC5.3 & 1.5.3.  & Usunięcie czasopisma lub zasobu literaturowego z systemu & Użytkownik usuwa informacje dotyczące czasopisma lub zasobu literaturowego z systemu \\
	\hline
\end{tabular}
\end{table}

\begin{table}[!ht]
	\begin{center}
		{\Large\bf Współpraca z modułem rejestru pracowników} \\
	\end{center}
\begin{tabular}{| c | c | p{.20\textwidth} | p{.40\textwidth} |}
	\hline \textbf{ID} & \textbf{Wymaganie} & \textbf{Nazwa} & \textbf{Opis} \\
	\hline UC6.1 & 1.6.1. & Zapisanie informacji o osobie odpowiedzialnej za zasób & Użytkownik zapisuje informację o osobie odpowiedzialnej za konkretny zasób, wybierając ją z listy pracowników - za zasób odpowiedzialna jest dokładnie jednen pracownik \\
	\hline UC6.2 & 1.6.2. & Zapisanie informacji o użytkowniku konkretnego zasobu & Użytkownik zapisuje informację o użytkowniku konkretnego zasobu,  wybierając go z listy pracowników - dany zasób może nie być użytkowany przez żadną osobe lub może być użytkowany przez dokładnie jedną lub kilku użytkowników (w zależności od rodzaju zasobu) \\
	\hline
\end{tabular}
\end{table}

\begin{table}[!ht]
	\begin{center}
		{\Large\bf Obliczanie statystyk} \\
	\end{center}
\begin{tabular}{| c | c | p{.20\textwidth} | p{.40\textwidth} |}
	\hline \textbf{ID} & \textbf{Wymaganie} & \textbf{Nazwa} & \textbf{Opis} \\
	\hline UC7.1 & 1.7.1. & Prezentacja zakupów konkretnych zasobów w poszczególnych latach & Użytkownik ma możliwość zobaczenia statystyk dot. zakupów konkretnych zasobów w poszczególnych latach (podział możliwy zarówno na całą kategorie zasobu jak i jeden lub kilka zasobów z danej kategorii) \\
	\hline UC7.2.1 & 1.7.2. & Prezentacja ilości zakupionych zasobów w poszczególnych działach & Użytkownik ma możliwość zobaczenia statystyk dot. zakupionych zasobów w poszczególnych działach (podział możliwy zarówno na całą kategorie zasobu jak i jeden lub kilka zasobów z danej kategorii)\\
	\hline UC7.2.2. & 1.7.3. & Prezentacja ilości używanych zasobów w poszczególnych działach & Użytkownik ma możliwość zobaczenia statystyk dot. używanych zasobów w poszczególnych działach (podział możliwy zarówno na całą kategorie zasobu jak i jeden lub kilka zasobów z danej kategorii)\\
	\hline UC7.2.3 & 1.7.4. & Prezentacja ilości napraw zasobów w poszczególnych działach & Użytkownik ma możliwość zobaczenia statystyk dot. napraw zasobów w poszczególnych działach (podział możliwy zarówno na całą kategorie zasobu jak i jeden lub kilka zasobów z danej kategorii)\\
	\hline UC7.3.1 & 1.7.5. & Prezentacja ilości zakupionych zasobów w poszczególnych placówkach & Użytkownik ma możliwość zobaczenia statystyk dot. zakupionych zasobów w poszczególnych placówkach (podział możliwy zarówno na całą kategorie zasobu jak i jeden lub kilka zasobów z danej kategorii)\\
	\hline UC7.3.2. & 1.7.6. & Prezentacja ilości używanych zasobów w poszczególnych placówkach & Użytkownik ma możliwość zobaczenia statystyk dot. używanych zasobów w poszczególnych placówkach (podział możliwy zarówno na całą kategorie zasobu jak i jeden lub kilka zasobów z danej kategorii)\\
	\hline UC7.3.3 & 1.7.7. & Prezentacja ilości napraw zasobów w poszczególnych placówkach & Użytkownik ma możliwość zobaczenia statystyk dot. napraw zasobów w poszczególnych placówkach (podział możliwy zarówno na całą kategorie zasobu jak i jeden lub kilka zasobów z danej kategorii)\\
	\hline
\end{tabular}
\end{table}

\begin{table}[!ht]
	\begin{center}
		{\Large\bf Rejestracja informacji dotyczącej miejsca zakupu} \\
	\end{center}
\begin{tabular}{| c | c | p{.20\textwidth} | p{.40\textwidth} |}
	\hline \textbf{ID} & \textbf{Wymaganie} & \textbf{Nazwa} & \textbf{Opis} \\
	\hline UC8 & 2.1. & Rejestracja informacji o miejscu zakupu zasobu & Użytkownik rejestruje miejsce zakupu konkretnego zasobu \\
	\hline
\end{tabular}
\end{table}

\begin{table}[!ht]
	\begin{center}
		{\Large\bf Historia napraw sprzętu} \\
	\end{center}
\begin{tabular}{| c | c | p{.20\textwidth} | p{.40\textwidth} |}
	\hline \textbf{ID} & \textbf{Wymaganie} & \textbf{Nazwa} & \textbf{Opis} \\
	\hline UC9.1 & 2.2.1. & Rejestracja informacji o wewnętrznej naprawie sprzętu & Użytkownik rejestruje fakt przeprowadzenia wewnętrznej naprawy sprzętu, wybierając pracownika, który przeprowadził naprawę z listy pracowników \\
	\hline UC9.2 & 2.2.2. & Rejestracja informacji o zewnętrznej naprawie sprzętu & Użytkownik rejestruje fakt przeprowadzenia zewnętrznej naprawy sprzętu \\
	\hline
\end{tabular}
\end{table}

\begin{table}[!ht]
	\begin{center}
		{\Large\bf Rejestracja aktualizacji oprogramowania} \\
	\end{center}
\begin{tabular}{| c | c | p{.20\textwidth} | p{.40\textwidth} |}
	\hline \textbf{ID} & \textbf{Wymaganie} & \textbf{Nazwa} & \textbf{Opis} \\
	\hline UC10 & 2.3.1. & Rejestracja aktualizacji oprogramowania & Użytkownik rejestruje przeprowadzenie aktualizacji oprogramowania \\
	\hline
\end{tabular}
\end{table}

\begin{table}[!ht]
	\begin{center}
		{\Large\bf Rejestracja operacji serwisowej} \\
	\end{center}
\begin{tabular}{| c | c | p{.20\textwidth} | p{.40\textwidth} |}
	\hline \textbf{ID} & \textbf{Wymaganie} & \textbf{Nazwa} & \textbf{Opis} \\
	\hline UC11 & 2.4.1. & Rejestracja operacji serwisowej & Użytkownik rejestruje przeprowadzenie operacji serwisowej zasobu \\
	\hline
\end{tabular}
\end{table}


\section{Specyfikacja przypadków użycia - poziom rozszerzony}
\newcommand{\myparagraph}[1]{\paragraph{#1}\mbox{}\\}

\subsection{Aktorzy}
\begin{itemize}
\item Użytkownik systemu
\end{itemize}


\subsection{PU1 Wyszukiwanie zasobów} \label{pu1}
\myparagraph{Opis}
Przypadek wyszukiwania zasobów za pomocą zadanych kryteriów przez użytkownika.

\myparagraph{Aktorzy}
Użytkownik systemu.

\myparagraph{Warunki wstępne}
\begin{itemize}
\item Użytkownik jest zalogowany w systemie.
\end{itemize}

\myparagraph{Warunki końcowe}
Przypadek użycia nie wpływa na stan systemu.

\myparagraph{Przebieg podstawowy}
\begin{enumerate}
\item \label{pu1:f} Aktor wybiera ikonę wyszukiwania zasobów.
\item \label{pu1:s}System prezentuje okno wyszukiwania zasobów zawierające sekcję podawania kryteriów.
\item Aktor wprowadza kryteria wyszukiwania.
\item Aktor wciska przycisk ,,Wyszukaj''.
\item System prezentuje wyniki wyszukiwania.
\end{enumerate}

\myparagraph{Alternatywne przebiegi zdarzeń}
\begin{enumerate}
\item Aktor nie podał kryteriów wyszukiwania
	\begin{enumerate}[label*=\arabic*.]
	\item Kroki \ref{pu1:f} - \ref{pu1:s} jak w przebiegu podstawowym.
	\item Aktor wciska przycisk ,,Wyszukaj''.
	\item System wyświetla informację o konieczności podania kryteriów wyszukiwania.
	\end{enumerate}
\end{enumerate}

\myparagraph{Sytuacje wyjątkowe}
Brak połączenia z siecią.



\subsection{PU2 Obsługa sprzetowego}

\subsubsection{PU2.1 Dodanie nowego sprzętu komputerowego do systemu}

\myparagraph{Opis}
Przypadek dodawania nowego sprzętu komputerowego do systemu katalogowego.

\myparagraph{Aktorzy}
Użytkownik systemu.

\myparagraph{Warunki wstępne}
\begin{itemize}
\item Użytkownik jest zalogowany w systemie.
\end{itemize}

\myparagraph{Warunki końcowe}
\begin{itemize}
\item Do systemu katalogowego został dodany nowy sprzęt komputerowy.
\end{itemize}

\myparagraph{Przebieg podstawowy}
\begin{enumerate}
\item \label{pu2.1:1} Aktor wybiera ikonę dodawania sprzętu komputerowego.
\item System prezentuje okno dodawania sprzętu komputerowego.
\item Aktor podaje dane dodawanego zasobu.
\item \label{pu2.1:4} Aktor wciska przycisk ,,Dodaj''.
\item System wyświetla potwierdzenie dodania sprzętu komputerowego do systemu.
\end{enumerate}

\myparagraph{Alternatywne przebiegi zdarzeń}
\begin{enumerate}
\item Aktor nie posiada wystarczających uprawnień do dodania sprzętu komputerowego.
	\begin{enumerate}[label*=\arabic*.]
		\item Krok \ref{pu2.1:1} jak w przebiegu podstawowym.
		\item System wyświetla informację o brak wymaganych uprawnień.
	\end{enumerate}
\item Aktor podał niepoprawne lub niepełne dane sprzętu komputerowego.
	\begin{enumerate}[label*=\arabic*.]
		\item Kroki \ref{pu2.1:1} - \ref{pu2.1:4} jak w przebiegu podstawowym.
		\item System wyświetla informację o konieczności poprawienia danych.
	\end{enumerate}
\end{enumerate}

\myparagraph{Sytuacje wyjątkowe}
Brak połączenia z siecią.



\subsubsection{PU2.2 Edycja informacji o sprzęcie komputerowym}

\myparagraph{Opis}
Przypadek edycji danych sprzętu komputerowego znajdującego się w systemie katalogowym.

\myparagraph{Aktorzy}
Użytkownik systemu.

\myparagraph{Warunki wstępne}
\begin{itemize}
\item Użytkownik jest zalogowany w systemie.
\item Sprzęt komputerowy istnieje w systemie katalogowym.
\end{itemize}

\myparagraph{Warunki końcowe}
\begin{itemize}
\item Dane sprzętu komputerowego zostały zmienione.
\end{itemize}

\myparagraph{Przebieg podstawowy}
\begin{enumerate}
\item \label{pu2.2:1} Aktor wyszukuje sprzęt zgodnie z \ref{pu1}
\item \label{pu2.2:2} Aktor wybiera sprzęt do edycji oraz wciska przycisk ,,Edytuj''.
\item System prezentuje okno edycji danych sprzętu komputerowego.
\item Aktor wprowadza nowe dane.
\item \label{pu2.2:5} Aktor wciska przycisk ,,Zapisz''.
\item System wyświetla potwierdzenie zmiany danych sprzętu komputerowego.
\end{enumerate}

\myparagraph{Alternatywne przebiegi zdarzeń}
\begin{enumerate}
\item Aktor nie posiada uprawnień do edycji danych sprzętu komputerowego.
	\begin{enumerate}[label*=\arabic*.]
		\item Kroki \ref{pu2.2:1} - \ref{pu2.2:2} jak w przebiegu podstawowym.
		\item System wyświetla informację o braku uprawnień do edycji.
	\end{enumerate}
\item Aktor podał niepoprawne lub niepełne dane sprzętu komputerowego.
	\begin{enumerate}[label*=\arabic*.]
		\item Kroki \ref{pu2.2:1} - \ref{pu2.2:5} jak w przebiegu podstawowym.
		\item System wyświetla informację o konieczności poprawienia danych.
	\end{enumerate}
\end{enumerate}

\myparagraph{Sytuacje wyjątkowe}\
Brak połączenia z siecią.

\subsubsection{PU2.3 Usunięcie sprzętu komputerowego z systemu}

\myparagraph{Opis}
Przypadek użycia opisuje procedurę usunięcia sprzętu komputerowego z systemu katalogowego.

\myparagraph{Aktorzy}
Użytkownik systemu.

\myparagraph{Warunki wstępne}
\begin{itemize}
\item Użytkownik jest zalogowany w systemie.
\item Sprzęt komputerowy istnieje w systemie katalogowym.
\end{itemize}

\myparagraph{Warunki końcowe}
\begin{itemize}
\item Sprzęt komputerowy został usunięty z systemu katalogowego.
\end{itemize}

\myparagraph{Przebieg podstawowy}
\begin{enumerate}
\item \label{pu2.3:1} Aktor wyszukuje sprzęt zgodnie z \ref{pu1}
\item \label{pu2.3:2} Aktor wybiera sprzęt do usunięcia oraz wciska przycisk ,,Usuń''.
\item System wyświetla okno z prośbą o potwierdzenie wykonania operacji.
\item Aktor potwierdza wykonanie operacji.
\item System wyświetla potwierdzenie usunięcia sprzętu komputerowego z systemu.
\end{enumerate}

\myparagraph{Alternatywne przebiegi zdarzeń}
\begin{enumerate}
\item Aktor nie posiada uprawnień do usunięcia sprzętu komputerowego z systemu.
	\begin{enumerate}[label*=\arabic*.]
		\item Kroki \ref{pu2.3:1} - \ref{pu2.3:2} jak w przebiegu podstawowym.
		\item System wyświetla informację o braku uprawnień do usunięcia sprzętu komputerowego.
	\end{enumerate}
\end{enumerate}

\myparagraph{Sytuacje wyjątkowe}\
Brak połączenia z siecią.

\subsubsection{PU3.1 Dodanie nowego oprogramowania do systemu}

\myparagraph{Opis}
Przypadek użycia opisuje procedurę dodania nowego oprogramowania do systemu katalogowego.

\myparagraph{Aktorzy}
Użytkownik systemu.

\myparagraph{Warunki wstępne}
\begin{itemize}
\item Użytkownik jest zalogowany w systemie.
\end{itemize}

\myparagraph{Warunki końcowe}
\begin{itemize}
\item Oprogramowanie (wraz z danymi instalacji) zostało zapisane w systemie katalogowym.
\end{itemize}

\myparagraph{Przebieg podstawowy}
\begin{enumerate}
\item \label{pu3.1:1} Aktor wybiera ikonę dodawania oprogramowania.
\item System prezentuje okno dodawania oprogramowania.
\item Aktor podaje dane dodawanego zasobu wraz z informacjami na temat przeprowadzonych instalacji.
\item \label{pu3.1:4} Aktor wciska przycisk ,,Dodaj''.
\item System wyświetla potwierdzenie dodania oprogramowania do systemu katalogowego.
\end{enumerate}

\myparagraph{Alternatywne przebiegi zdarzeń}
\begin{enumerate}
\item Aktor nie posiada uprawnień do dodania oprogramowania.
	\begin{enumerate}[label*=\arabic*.]
		\item Krok \ref{pu3.1:1} jak w przebiegu podstawowym.
		\item System wyświetla informację o braku uprawnień do wykonania operacji dodania oprogramowania.
	\end{enumerate}
\item Aktor podał niepoprawne lub niepełne dane.
	\begin{enumerate}[label*=\arabic*.]
		\item Kroki \ref{pu3.1:1} - \ref{pu3.1:4} jak w przebiegu podstawowym.
		\item System wyświetla informację o konieczności poprawienia danych.
	\end{enumerate}
\end{enumerate}

\myparagraph{Sytuacje wyjątkowe}
Brak połączenia z siecią.

\subsubsection{PU3.2 Edycja informacji oprogramowaniu}

\myparagraph{Opis}
Przypadek użycia opisuje procedurę edycji informacji dotyczących oprogramowania znajdującego się w systemie - możliwa jest zmiana podstawowych danych oraz edycja przeprowadzonych instalacji.

\myparagraph{Aktorzy}
Użytkownik systemu.

\myparagraph{Warunki wstępne}
\begin{itemize}
\item Użytkownik jest zalogowany w systemie.
\item Oprogramowanie istnieje w systemie katalogowym.
\end{itemize}

\myparagraph{Warunki końcowe}
\begin{itemize}
\item Dane oprogramowania zostały zmienione.
\end{itemize}

\myparagraph{Przebieg podstawowy}
\begin{enumerate}
\item \label{pu3.2:1} Aktor wyszukuje oprogramowanie zgodnie z \ref{pu1}
\item \label{pu3.2:2} Aktor wybiera oprogramowanie do edycji oraz wciska przycisk ,,Edytuj''.
\item System prezentuje okno edycji danych oprogramowania.
\item Aktor wprowadza nowe dane.
\item \label{pu3.2:5} Aktor wciska przycisk ,,Zapisz''.
\item System wyświetla potwierdzenie zmiany danych oprogramowania.
\end{enumerate}

\myparagraph{Alternatywne przebiegi zdarzeń}
\begin{enumerate}
\item Aktor nie posiada uprawnień do edycji danych oprogramowania.
	\begin{enumerate}[label*=\arabic*.]
		\item Kroki \ref{pu3.2:1} - \ref{pu3.2:2} jak w przebiegu podstawowym.
		\item System wyświetla informację o braku uprawnień do edycji.
	\end{enumerate}
\item Aktor podał niepoprawne lub niepełne dane oprogramowania.
	\begin{enumerate}[label*=\arabic*.]
		\item Kroki \ref{pu3.2:1} - \ref{pu3.2:5} jak w przebiegu podstawowym.
		\item System wyświetla informację o konieczności poprawienia danych.
	\end{enumerate}
\end{enumerate}

\myparagraph{Sytuacje wyjątkowe}\
Brak połączenia z siecią.

\subsubsection{PU3.3 Usunięcie oprogramowania z systemu}

\myparagraph{Opis}
Przypadek użycia opisuje procedurę usunięcia oprogramowania (wraz z danymi na temat przeprowadzonych instalacji) z systemu katalogowego.

\myparagraph{Aktorzy}
Użytkownik systemu.

\myparagraph{Warunki wstępne}
\begin{itemize}
\item Użytkownik jest zalogowany w systemie.
\item Oprogramowanie istnieje w systemie katalogowym.
\end{itemize}

\myparagraph{Warunki końcowe}
\begin{itemize}
\item Oprogramowanie zostało usunięte z systemu katalogowego.
\end{itemize}

\myparagraph{Przebieg podstawowy}
\begin{enumerate}
\item \label{pu3.3:1} Aktor wyszukuje sprzęt zgodnie z \ref{pu1}
\item \label{pu3.3:2} Aktor wybiera oprogramowanie oraz wciska przycisk ,,Usuń''.
\item System wyświetla okno z prośbą o potwierdzenie wykonania operacji.
\item Aktor potwierdza wykonanie operacji.
\item System wyświetla potwierdzenie usunięcia oprogramowania z systemu.
\end{enumerate}

\myparagraph{Alternatywne przebiegi zdarzeń}
\begin{enumerate}
\item Aktor nie posiada uprawnień do usunięcia oprogramowania z systemu.
	\begin{enumerate}[label*=\arabic*.]
		\item Kroki \ref{pu3.3:1} - \ref{pu3.3:2} jak w przebiegu podstawowym.
		\item System wyświetla informację o braku uprawnień do usunięcia oprogramowania.
	\end{enumerate}
\end{enumerate}

\myparagraph{Sytuacje wyjątkowe}\
Brak połączenia z siecią.

\subsubsection{PU4.1 Dodanie innego sprzętu, urządzeń lub wyposażenia do systemu}

\myparagraph{Opis}
Przypadek użycia opisuje procedurę dodania sprzętu, urządzeń lub wyposażenia do systemu.

\myparagraph{Aktorzy}
Użytkownik systemu.

\myparagraph{Warunki wstępne}
\begin{itemize}
\item Użytkownik jest zalogowany w systemie.
\end{itemize}

\myparagraph{Warunki końcowe}
\begin{itemize}
\item Zasób (wraz z danymi instalacji) został zapisany w systemie katalogowym.
\end{itemize}

\myparagraph{Przebieg podstawowy}
\begin{enumerate}
\item \label{pu4.1:1} Aktor wybiera ikonę dodawania zasobu.
\item System prezentuje okno dodawania zasobu.
\item Aktor podaje dane dodawanego zasobu.
\item \label{pu4.1:4} Aktor wciska przycisk ,,Dodaj''.
\item System wyświetla potwierdzenie dodania zasobu do systemu katalogowego.
\end{enumerate}

\myparagraph{Alternatywne przebiegi zdarzeń}
\begin{enumerate}
\item Aktor nie posiada uprawnień do dodania zasobu.
	\begin{enumerate}[label*=\arabic*.]
		\item Krok \ref{pu4.1:1} jak w przebiegu podstawowym.
		\item System wyświetla informację o braku uprawnień do wykonania operacji dodania zasobu.
	\end{enumerate}
\item Aktor podał niepoprawne lub niepełne dane.
	\begin{enumerate}[label*=\arabic*.]
		\item Kroki \ref{pu4.1:1} - \ref{pu4.1:4} jak w przebiegu podstawowym.
		\item System wyświetla informację o konieczności poprawienia danych.
	\end{enumerate}
\end{enumerate}

\myparagraph{Sytuacje wyjątkowe}
Brak połączenia z siecią.

\subsubsection{PU4.2 Edycja informacji o innym sprzęcie, urządzeniu lub wyposażeniu}

\myparagraph{Opis}
Przypadek użycia opisuje procedurę edycji informacji dotyczących ogólnych zasobów.

\myparagraph{Aktorzy}
Użytkownik systemu.

\myparagraph{Warunki wstępne}
\begin{itemize}
\item Użytkownik jest zalogowany w systemie.
\item Zasób istnieje w systemie katalogowym.
\end{itemize}

\myparagraph{Warunki końcowe}
\begin{itemize}
\item Dane zasobu zostały zmienione.
\end{itemize}

\myparagraph{Przebieg podstawowy}
\begin{enumerate}
\item \label{pu4.2:1} Aktor wyszukuje zasób zgodnie z \ref{pu1}
\item \label{pu4.2:2} Aktor wybiera zasób do edycji oraz wciska przycisk ,,Edytuj''.
\item System prezentuje okno edycji danych zasobu.
\item Aktor wprowadza nowe dane.
\item \label{pu4.2:5} Aktor wciska przycisk ,,Zapisz''.
\item System wyświetla potwierdzenie zmiany danych zasobu.
\end{enumerate}

\myparagraph{Alternatywne przebiegi zdarzeń}
\begin{enumerate}
\item Aktor nie posiada uprawnień do edycji danych zasobu.
	\begin{enumerate}[label*=\arabic*.]
		\item Kroki \ref{pu4.2:1} - \ref{pu4.2:2} jak w przebiegu podstawowym.
		\item System wyświetla informację o braku uprawnień do edycji.
	\end{enumerate}
\item Aktor podał niepoprawne lub niepełne dane oprogramowania.
	\begin{enumerate}[label*=\arabic*.]
		\item Kroki \ref{pu4.2:1} - \ref{pu4.2:5} jak w przebiegu podstawowym.
		\item System wyświetla informację o konieczności poprawienia danych.
	\end{enumerate}
\end{enumerate}

\myparagraph{Sytuacje wyjątkowe}\
Brak połączenia z siecią.

\subsubsection{PU4.3 Usunięcie innego sprzętu, urządzenia lub wyposażenia z systemu}

\myparagraph{Opis}
Przypadek użycia opisuje procedurę usunięcia zasobu (ogólnego typu) z systemu katalogowego.

\myparagraph{Aktorzy}
Użytkownik systemu.

\myparagraph{Warunki wstępne}
\begin{itemize}
\item Użytkownik jest zalogowany w systemie.
\item Zasób istnieje w systemie katalogowym.
\end{itemize}

\myparagraph{Warunki końcowe}
\begin{itemize}
\item Zasób został usunięty z systemu katalogowego.
\end{itemize}

\myparagraph{Przebieg podstawowy}
\begin{enumerate}
\item \label{pu4.3:1} Aktor wyszukuje zasób zgodnie z \ref{pu1}
\item \label{pu4.3:2} Aktor wybiera zasób oraz wciska przycisk ,,Usuń''.
\item System wyświetla okno z prośbą o potwierdzenie wykonania operacji.
\item Aktor potwierdza wykonanie operacji.
\item System wyświetla potwierdzenie usunięcia zasobu z systemu.
\end{enumerate}

\myparagraph{Alternatywne przebiegi zdarzeń}
\begin{enumerate}
\item Aktor nie posiada uprawnień do usunięcia zasobu z systemu.
	\begin{enumerate}[label*=\arabic*.]
		\item Kroki \ref{pu4.3:1} - \ref{pu4.3:2} jak w przebiegu podstawowym.
		\item System wyświetla informację o braku uprawnień do usunięcia zasobu.
	\end{enumerate}
\end{enumerate}

\myparagraph{Sytuacje wyjątkowe}\
Brak połączenia z siecią.

\subsubsection{PU5.1 Dodanie nowego czasopisma lub zasobu literaturowego do systemu}

\myparagraph{Opis}
Przypadek użycia opisuje procedurę dodania czasopisma bądź zasobu literaturowego do systemu.

\myparagraph{Aktorzy}
Użytkownik systemu.

\myparagraph{Warunki wstępne}
\begin{itemize}
\item Użytkownik jest zalogowany w systemie.
\end{itemize}

\myparagraph{Warunki końcowe}
\begin{itemize}
\item Dane czasopisma lub zasobu literackiego zostały zapisane w systemie katalogowym.
\end{itemize}

\myparagraph{Przebieg podstawowy}
\begin{enumerate}
\item \label{pu5.1:1} Aktor wybiera ikonę dodawania czasopisma/zasobu literaturowego.
\item System prezentuje okno dodawania zasobu.
\item Aktor podaje dane dodawanego zasobu.
\item \label{pu5.1:4} Aktor wciska przycisk ,,Dodaj''.
\item System wyświetla potwierdzenie dodania zasobu do systemu katalogowego.
\end{enumerate}

\myparagraph{Alternatywne przebiegi zdarzeń}
\begin{enumerate}
\item Aktor nie posiada uprawnień do dodania zasobu.
	\begin{enumerate}[label*=\arabic*.]
		\item Krok \ref{pu5.1:1} jak w przebiegu podstawowym.
		\item System wyświetla informację o braku uprawnień do wykonania operacji dodania zasobu.
	\end{enumerate}
\item Aktor podał niepoprawne lub niepełne dane.
	\begin{enumerate}[label*=\arabic*.]
		\item Kroki \ref{pu5.1:1} - \ref{pu5.1:4} jak w przebiegu podstawowym.
		\item System wyświetla informację o konieczności poprawienia danych.
	\end{enumerate}
\end{enumerate}

\myparagraph{Sytuacje wyjątkowe}
Brak połączenia z siecią.

\subsubsection{PU5.2 Edycja informacji o czasopiśmie lub zasobie literaturowym}

\myparagraph{Opis}
Przypadek użycia opisuje procedurę edycji informacji dotyczących czasopisma bądź zasobu literaturowego.

\myparagraph{Aktorzy}
Użytkownik systemu.

\myparagraph{Warunki wstępne}
\begin{itemize}
\item Użytkownik jest zalogowany w systemie.
\item Czasopismo/zasób literaturowy istnieje w systemie katalogowym.
\end{itemize}

\myparagraph{Warunki końcowe}
\begin{itemize}
\item Dane zasobu zostały zmienione.
\end{itemize}

\myparagraph{Przebieg podstawowy}
\begin{enumerate}
\item \label{pu5.2:1} Aktor wyszukuje zasób zgodnie z \ref{pu1}
\item \label{pu5.2:2} Aktor wybiera zasób do edycji oraz wciska przycisk ,,Edytuj''.
\item System prezentuje okno edycji danych zasobu.
\item Aktor wprowadza nowe dane.
\item \label{pu5.2:5} Aktor wciska przycisk ,,Zapisz''.
\item System wyświetla potwierdzenie zmiany danych zasobu.
\end{enumerate}

\myparagraph{Alternatywne przebiegi zdarzeń}
\begin{enumerate}
\item Aktor nie posiada uprawnień do edycji danych zasobu.
	\begin{enumerate}[label*=\arabic*.]
		\item Kroki \ref{pu5.2:1} - \ref{pu5.2:2} jak w przebiegu podstawowym.
		\item System wyświetla informację o braku uprawnień do edycji.
	\end{enumerate}
\item Aktor podał niepoprawne lub niepełne dane oprogramowania.
	\begin{enumerate}[label*=\arabic*.]
		\item Kroki \ref{pu5.2:1} - \ref{pu5.2:5} jak w przebiegu podstawowym.
		\item System wyświetla informację o konieczności poprawienia danych.
	\end{enumerate}
\end{enumerate}

\myparagraph{Sytuacje wyjątkowe}\
Brak połączenia z siecią.

\subsubsection{PU5.3 Usunięcie czasopisma lub zasobu literaturowego z systemu}

\myparagraph{Opis}
Przypadek użycia opisuje procedurę usunięcia czasopisma/zasobu literaturowego z systemu katalogowego.

\myparagraph{Aktorzy}
Użytkownik systemu.

\myparagraph{Warunki wstępne}
\begin{itemize}
\item Użytkownik jest zalogowany w systemie.
\item Czasopismo/zasób literaturowy istnieje w systemie katalogowym.
\end{itemize}

\myparagraph{Warunki końcowe}
\begin{itemize}
\item Czasopismo/zasób literaturowy został usunięty z systemu katalogowego.
\end{itemize}

\myparagraph{Przebieg podstawowy}
\begin{enumerate}
\item \label{pu5.3:1} Aktor wyszukuje zasób zgodnie z \ref{pu1}
\item \label{pu5.3:2} Aktor wybiera zasób oraz wciska przycisk ,,Usuń''.
\item System wyświetla okno z prośbą o potwierdzenie wykonania operacji.
\item Aktor potwierdza wykonanie operacji.
\item System wyświetla potwierdzenie usunięcia zasobu z systemu.
\end{enumerate}

\myparagraph{Alternatywne przebiegi zdarzeń}
\begin{enumerate}
\item Aktor nie posiada uprawnień do usunięcia zasobu z systemu.
	\begin{enumerate}[label*=\arabic*.]
		\item Kroki \ref{pu5.3:1} - \ref{pu5.3:2} jak w przebiegu podstawowym.
		\item System wyświetla informację o braku uprawnień do usunięcia zasobu.
	\end{enumerate}
\end{enumerate}

\myparagraph{Sytuacje wyjątkowe}\
Brak połączenia z siecią.
\section{Architektura systemu oraz oprogramowanie podstawowe}

Budowany system wspomagania zarządzania firmą informatyczną skłąda sie
z wielu modułów. Ponieważ niniejszy dokument stanowi dokumentację
projektową jedynie do jednego z modułów na rysunku \ref{fig:labelArchOgol}
zaprezentowano ogólny diagram komponentów na które podzielony jest system.

\begin{figure}[h]
	\centering
	\includegraphics[width=\textwidth]{img/archogol}
	\caption{Diagram komponentów całego systemu\label{fig:labelArchOgol}}
\end{figure}

Bardzo istotnym komponentem systemu jest komponent
komunikacyjny. Został on opisany w osobnym dokumencie wraz ze
szczególami jego interfejsu. Wszystkie pozostałe moduły nie
dostarczają własnych interfejsów lecz korzystają z generycznego
interfejsu dostarczanego przez wspomniany moduł.

Przechodząc zatem do perspektywy omawianego modułu rejestru dostępnych
zasobów system zostanie zrealizowany w architekturze trójwarstwowej z
wykorzystaniem modelu klient-serwer. System będzie składał się z
następujących warstw:

\begin{itemize}
	\item[--] warstwa prezentacji (interfejs użytkownika),
	\item[--] warstwa logiki biznesowej (pośrednicząca w dostępie do danych zapisanych w bazie danych),
	\item[--] warstwa danych.
\end{itemize}

\begin{figure}[h]
	\centering
	\includegraphics[width=0.8\textwidth]{img/components}
	\caption{Diagram komponentów \label{fig:labelComponents}}
\end{figure}


\subsection{Warstwa prezentacji}

Warstwa prezentacji będzie elementem odpowiedzialnym za prezentację
treści użytkownikowi końcowemu. Będzie również umożliwiać odbieranie
żądań użytkowników i przesyłanie ich do kolejnych warstw. Warstwa ta
zostanie zrealizowana w technologii cienkiego klienta. Dzięki
obecności warstwy prezentacji użytkownicy naszego systemu nie są
zobowiązani do przeprowadzania instalacji dodatkowego oprogramowania w
celu korzystania z jego możliwości - jedynym wymogiem jest posiadanie
przeglądarki internetowej, za pomocą której użytkownik będzie logował
się do systemu. Aby umożliwić prezentację treści użytkownikowi w
przejrzysty sposób przeglądarka po stronie klienta pozwinna obsługiwać
szablony CSS, technologię HTML5 oraz język JavaScript.

\subsection{Warstwa logiki biznesowej}

Warstwa logiki biznesowej będzie składać się ze zbioru komponentów
odpowiedzialnych za spełnianie poszczególnych założeń funkcjonalnych
systemu (o których mowa w niniejszym dokumencie). Zostanie ona
zaimplementowana przy użyciu technologii JavaEE, a jako serwer
aplikacyjny zostanie wykorzystany JBoss Application Server. Do
komunikacji z baza danych wykorzystany zostanie interfejs Java
Hibernate. Całość zostanie zintegrowana z Apache Web Server.

Warstwa aplikacji stanowi najważniejszy element całego systemu i jest
odpowiedzialna za całą logikę zawartą w aplikacji. W związku z
powyższym jest ona dość skomplikowana i składa się z wielu elementów.

\subsection{Warstwa danych}

Warstwa danych odpowiedzialna będzie za dostęp do bazy danych -
odczytywanie i zapisywanie w niej informacji. Do realizacji projektu
wybrano relacyjną bazę danych SQL. Jako system zarządzania bazą danych
zostanie wykorzystany będzie Oracle Database 12c w wersji
Enterprise. W celu zapewnienia trwałości przechowywanych danych oraz
wysokiej dostępności systemu w warstwie tej funkcjonowały będą dwie
instancje bazy danych - podstawowa i zapasowa.


\section{Specyfikacja sprzętu oraz oprogramowania podstawowego}

System zostanie zrealizowany w architekturze trójwarstwowej. Będzie składał się z następujących warstw:

\begin{itemize}
	\item[--] warstwa prezentacji (interfejs użytkownika),
	\item[--] warstwa logiki biznesowej (pośrednicząca w dostępie do danych zapisanych w bazie danych),
	\item[--] warstwa danych.
\end{itemize}

\subsection{Warstwa prezentacji}

Warstwa prezentacji będzie elementem odpowiedzialnym za prezentację treści użytkownikowi końcowemu. Będzie również umożliwiać odbieranie żądań użytkowników i przesyłanie ich do kolejnych warstw. Dzięki obecności warstwy prezentacji użytkownicy naszego systemu nie są zobowiązani do przeprowadzania instalacji dodatkowego oprogramowania w celu korzystania z jego możliwości - jedynym wymogiem jest posiadanie przeglądarki internetowej, za pomocą której użytkownik będzie logował się do systemu.

\subsection{Warstwa logiki biznesowej}

Warstwa logiki biznesowej będzie składać się ze zbioru komponentów odpowiedzialnych za
spełnianie poszczególnych założeń funkcjonalnych systemu (o których mowa w niniejszym dokumencie).

\subsection{Warstwa danych}

Warstwa danych odpowiedzialna będzie za dostęp do bazy danych - odczytywanie i zapisywanie w niej informacji.

\begin{figure}[H]
	\centering
	\includegraphics[scale=0.4]{img/architektura}
	\caption{Schemat architektury \label{fig:labelArchitecture}}
\end{figure}

\subsection{Schemat architektury}

Jak uwidoczniono na \ref*{fig:labelArchitecture} na wejściu żądania będą obsługiwany przez router główny, któremu towarzyszył będzie router zapasowy w przypadku, kiedy router główny nie będzie w stanie obsłużyć żądania. Następnie wybierany będzie Web Server, który będzie obsługiwał zgłoszenie (w zależności od poziomu obciążenia serwerów, decydować o tym będzie Load Balancer). W analogiczny sposób wybierany będzie serwer aplikacyjny, pośredniczący w obsłudze.

W odniesieniu do serwerów bazy danych zastosowany będzie mechanizm replikacji danych, który ma zagwarantować bezpieczeństwo przechowywanych informacji (ochronę przed ich utratą w wyniku awarii jednego z serwerów).

\subsection{Oszacowanie rozmiarów systemu}

W chwili obecnej firma posiada w ewidencji:

\begin{itemize}
\item[--] około 50 pojazdów
\item[--] około 800 telefonów
\item[--] około 1500 komputerów stacjonarnych
\item[--] około 2500 monitorów komputerowych
\item[--] około 2000 laptopów
\item[--] około 100 serwerów
\item[--] około 200 sztuk sprzętu serwerowego
\item[--] około 10000 sztuk innego drobnego sprzętu komputerowego
\item[--] około 300 nośników z oprogramowaniem
\item[--] około 200 licencji na użytkowanie oprogramowania (jedna licencja na wiele stanowisk)
\end{itemize}

Oznacza to, że w ewidencji firmy jest ponad 17 tysięcy przedmiotów i
licencji na oprogramowanie. Ponieważ firma się intensywnie rozwija
należy założyć, że system będzie obsługiwał co najmniej 30 tysięcy
przedmiotów. W chwili obecnej firma zatrudnia około 1500
pracowników. Ponieważ firma przeżywa intensywny rozwój konieczne jest
zwymiarowanie systemu na poziomie co najmniej dwu krotności obecnego
zatrudnienia to jest 3000 pracowników. System należy do grupy systemów
wspierających funkcjonowanie przedsiębiorstwa i nie będzie
wykorzystywany w ramach podstawowych zadania pracowników. Należy zatem
założyć, że nie wszyscy pracownicy będą go używali w tym samym
czasie. W związku z tym należy założyć, że równocześnie system będzie
używany przez około 10\% pracowników czyli 300 osób. Powyższe szacunki
są zgodne z wymaganiami niefunkcjonalnymi przedstawionymi w
\ref{wymagania_niefunkcjonalne}.

\subsection{Rozkład obciążenia na poszczególne warstwy}

Aplikacja w architekturze trójwarstwowej posiada budowę bardzo
modularną, przez co możliwe jest rozmieszczenie różnych elementów na
odpowiednich maszynach. Dzięki takiemu rozmieszczeniu każda z warstw
może być wykonywana na sprzęcie dostosowanym do jej zadań. Ponieważ
każda warstwa wykonuje inne zadania to znacząco też różni się
charakterystyka sprzętu, który należy wykorzystać w danej warstwie.

\subsubsection{Warstwa prezentacji}

Warstwa prezentacji odpowiedzialna jest za wyświetlanie użytkownikowi
treści. Implementacja tej warstwy zostanie wykonana z użyciem języka
JavaScript wraz z technologiami HTML oraz CSS. Ponadto dodatkowa
funkcjonalność zostanie zaimplementowana przy użyciu biblioteki
jQuerry. Pomimo, iż zarówno serwer jak i klienci znajdować się będą w
sieci wewnętrznej przedsiębiorstwa należy zapewnić odpowiedni poziom
bezpieczeństwa poprzez wykorzystanie protokołu SSL/TLS.

Z perspektywy wymaganych zasobów sprzętowych wymagania dla tej warstwy
można podzielić na dwie części. Pierwsza z nich dotyczy wymagań co do
urządzenia na którym treść będzie prezentowana. Konieczne jest aby
takie urządzenie wyposażone było w przeglądarkę obsługującą
JavaScript. Druga część z wymagań sprzętowych dotyczy serwera www, do
którego kierowane będą żądania. Od tego urządzenia wymaga się przede
wszystkim wysokiej wydajności, krótkiego czasu odpowiedzi oraz
możliwości obsługi wielu klientów jednocześnie. Istotne jest również
zapewnienie wysokiej dostępności tego serwera.

\subsubsection{Warstwa logiki biznesowej}

Warstwa logiki biznesowej odpowiedzialna jest za przetwarzanie żądań
klienta w sposób zgodny z prawami rządzącymi organizacją. Zostanie ona
zaimplementowana w technologii J2EE. Jako serwer aplikacyjny wybrano
JBoos Application Server.

Z perspektywy wymaganych zasobów warstwa ta posiada analogiczne
wymagania jak warstwa prezentacji. Oznacza to, że istotna dostępność,
krótki czas odpowiedzi oraz możliwość obsługi wielu klientów
równocześnie. Te abstrakcyjne wymagania tej warstwy na sprzęt
przekładają się na duże zapotrzebowanie na procesor (w tym ich
liczbę), rozmiar pamięci RAM oraz szybkość odpowiedzi dysku twardego,
natomiast już sam rozmiar tego dysku może być nie zbyt duży.

\subsubsection{Warstwa danych}

Warstwa danych odpowiedzialna jest za przetwarzanie i przechowywanie
danych w sposób zapewniający ich spójność oraz trwałość. W tej
warstwie zostanie wykorzystana technologia Oracle DataBase 12c w
wersji Enterprise Edition. Jest to najnowsza wersja bazy danych od
Oracle i posiada wiele zaawansowanych mechanizmów, wpływających na
wydajność jak i bezpieczeństwo przechowywanych danych.

Z perspektywy wymaganych zasobów sprzętowych warstwa ta posiada
analogiczne wymagania co poprzednia lecz są one tutaj bardziej
krytyczne, gdyż zbyt długie przetwarzanie zapytania przez bazę danych
znacząco wydłuża czas odpowiedzi aplikacji. Należy również pamiętać o
dostosowaniu rozmiaru pamięci operacyjnej oraz zamontowaniu
odpowiedniej liczby wydajnych dysków twardych wraz z zapewnieniem ich
replikacji w celu zabezpieczenia się przed utrata danych.

\subsection{Sprzęt}

Budowa nowego systemu zarządzania przedmiotami i licencjami wymaga
znacznej rozbudowy dotychczasowej infrastruktury IT firmy.

\subsubsection{Warstwa prezentacji oraz aplikacji}

Część kliencka budowanego systemu będzie uruchamiana poprzez
przeglądarkę na komputerach pracowników firmy. Dzięki temu część
warstwy prezentacji, którą należy umieścić na serwerze składa się
wyłącznie z serwera WWW. Aby zatem ograniczyć konieczne zasoby
sprzętowe, a także opóźnienia w komunikacji pomiędzy warstwą
prezentacji, a warstwą aplikacji zdecydowano umieścić je na wspólnej
maszynie. Ponieważ klient ma wysokie wymagania co do dostępności
systemu, należy zakupić dwa serwery, które będą współdziałały i
dzieliły obciążenie pomiędzy siebie, a w razie awarii jednego z nich
drugi będzie w stanie zapewnić pełną funkcjonalność.


Jako serwer www oraz aplikacyjny powinna zostać wykorzystana maszyna,
która posiada wysoko wydajne procesory oraz pamięć RAM. Ważna jest
również możliwość montażu dysków twardych o krótkim czasie dostępu. W
związku z powyższym zdecydowano o wykorzystaniu serwerów Dell
PowerEdge R630.

\begin{figure}[H]
	\centering
	\includegraphics[width=\textwidth]{img/r630.jpg}
	\caption{Dell PowerEdge R630}
\end{figure}

Jest to wysoko wydajny serwer, który dzięki małych rozmiarów oszczędzi
miejsce w serwerowni oraz cechuje się niskim zużyciem
energii. Parametry techniczne serwera:

\begin{description}
\item[procesor] 2 sockety procesora (dla Intel® Xeon® E5 2600 v3)
\item[chipset] Intel C610
\item[pamięć RAM] do 768 GB (24 sloty DIMM) DDR4
\item[pamięć trwała] do 10 dysków (1,8 TB każdy)
\item[kontroler RAID] PERC H730P
\item[złącza I/O] do 3 złącz PCIe
\item[karta sieciowa] 4 x 1Gb, 2 x 1Gb + 2 x 10Gb, 4 x 10Gb
\item[obudowa] Rack 1U
\item[zasilanie] 2x 750W HotPlug
\end{description}

Aby serwer spełnił stawiane przed nim wymagania wydajnościowe należy
zakupić go w następującej konfiguracji:

\begin{description}
\item[procesor] 2x Intel® Xeon® Processor E5-2697 v3
\item[chipset] Intel C610
\item[pamięć RAM] 8x 16 GB DDR4
\item[pamięć trwała] 8x 146 GB 15k SAS (RAID 10)
\item[kontroler RAID] PERC H730P
\item[złącza I/O] do 3 złącz PCIe
\item[karta sieciowa] 4 x 1Gb, 2 x 1Gb + 2 x 10Gb, 4 x 10Gb
\item[obudowa] Rack 1U
\item[zasilanie] 2x 750W HotPlug
\end{description}

Zastosowanie dwóch serwerów o wskazanych parametrów pozwoli sprostać
wszystkim wymaganiom stawianym przez klienta. Ponadto zapewniona jest
możliwość skalowania dostarczonego rozwiązania poprzez dodanie
kolejnego serwera aplikacji lub rozbudowę (głównie pamięć operacyjna)
posiadanych już serwerów.

\subsubsection{Warstwa danych}

Serwer bazy danych powinien pozwalać na gromadzenie znacznej ilości
danych. Konieczne jest również zapewnienie redundantnego drugiego
serwera bazy danych, aby równoważyć obciążenie oraz zapewnić wysoką
dostępność. W celu zapewnienie odpowiednich możliwości w zakresie
przechowywania i przetwarzania dużej ilości danych postanowiono
wykożystać dwa serwery Dell PowerEdge R920.

\begin{figure}[H]
	\centering
	\includegraphics[width=\textwidth]{img/r920.jpg}
	\caption{Dell PowerEdge R920}
\end{figure}

Jest to wysokiej klasy serwer, którego parametry pozwolą na
przetwarzanie danych w czasie rzeczywistym. Parametry techniczne
serwera:

\begin{description}
\item[procesor] 4 sockety procesora (dla Intel Xeon E7-4800 v2 lub E7-8800 v2)
\item[chipset] Intel C602J
\item[pamięć RAM] do 6TB (96 slotów DIMM) DDR3L, RDIMM, LR-DIMM
\item[pamięć trwała] do 24 dysków (1,2 TB każdy)
\item[kontroler RAID] PERC H730P
\item[złącza I/O] do 10 złącz PCIe
\item[karta sieciowa] Intel Ethernet X540 10Gb BT DP + I350 1Gb BT DP
\item[karta graficzna] Matrox® G200 with 8MB memory
\item[obudowa] Rack 4U
\item[zasilanie] 4x 1100W HotPlug
\end{description}

Maksymalne parametry techniczne wybranego serwera znacząco
przewyższają bieżące zapotrzebowanie firmy dlatego wykorzystana
będzie następująca konfiguracja:

\begin{description}
\item[procesor] 2x Intel® Xeon® Processor E7-8880 v2 (37.5M Cache, 2.50 GHz)
\item[chipset] Intel C602J
\item[pamięć RAM] 8x 16 GB
\item[pamięć trwała] 2x 146 GB 15k SAS + 8x 1.2 TB
\item[kontroler RAID] PERC H730P
\item[złącza I/O] do 10 złącz PCIe
\item[karta sieciowa] Intel Ethernet X540 10Gb BT DP + I350 1Gb BT DP
\item[karta graficzna] Matrox® G200 with 8MB memory
\item[obudowa] Rack 4U
\item[zasilanie] 4x 1100W HotPlug
\end{description}

Zastosowana konfiguracja jest zdecydowanie wystarczająca dla
bieżących wymagań firmy, a zastosowanie wysokiej klasy serwera
pozostawia duże możliwości w zakresie skalowalności, dzięki czemu
serwery te będzie można dostosować do potrzeb intensywnie rozwijającej
się firmy. Dzięki wsparciu technologii wirtualizacji sprzętowej serwer
ten będzie mógł być efektywnie współdzielony z pozostałymi modułami
budowanego systemu.


\section{Analityczny model projektowy dziedziny}
\subsection{Diagram klas}
\begin{figure}[h!]
	\centering
	\includegraphics[scale=0.4]{img/class-diagram2}
	\caption{Diagram klas systemu.\label{fig:labelClassDiagram}}
\end{figure}

\newpage 

Opis klas przedstawionych na diagramie~\ref{fig:labelClassDiagram}:
\begin{description}
	\item \textbf{\textit{Person}} - klasa abstrakcyjna, będąca bazową dla klas \textit{User} oraz \textit{Administrator}. Przechowuje podstawowe informacje o osobie zawierające identyfikator, imię, nazwisko oraz adres email.
	
	\item \textbf{User} - specjalizacja klasy abstrakcyjnej \textit{Person} będąca użytkownikiem zasobu przechowywanego przez system.
	
	\item \textbf{Administrator} - specjalizacja klasy \textit{User} nadająca dodatkowe cechy polegające na umożliwieniu administrowania zasobami w systemie.
	
	\item \textbf{CompanyDepartment} - klasa słownikowa reprezentująca dział firmy. Komponuje osoby firmy.
	
	\item \textbf{Institution} - klasa reprezentująca oddział firmy. Komponuje działy firmy.
	
	\item \textbf{\textit{Resource}} - klasa abstrakcyjna reprezentująca byt w systemie. Jest generalizacją takich typów jak \textit{Software}, \textit{Hardware}, \textit{JournalOrLiterature} oraz \textit{DevicesAndOthers}. Zawiera informacje o identyfikatorze, roku produkcji, nazwa oraz opis.
	
	\item \textbf{Software} - klasa reprezentująca oprogramowanie. Jest specjalizacją klasy \textit{Resource}. Składa się z takich atrybutów jak wytwórca, wersja oraz numer licencji.
	
	\item \textbf{Hardware} - klasa reprezentująca sprzęt komputerowy. Jest specjalizacją klasy \textit{Resource}. Zawiera takie informacje jak numer seryjny, konfigurację, parametry oraz potencjalne zestawy w skład których może wchodzić.
		
	\item \textbf{JournalOrLiterature} - klasa reprezentująca czasopisma oraz literaturę. Jest specjalizacją klasy \textit{Resource}. W skład jej atrybutów wchodzą numer ISBN, wydawca, autor oraz tytuł.
		
	\item \textbf{DevicesAndOthers} - klasa reprezentująca inny sprzęt taki jak samochody, telefony itp (zdefiniowane w enumeracji \textit{PossibleType}). Jest specjalizacją klasy \textit{Resource}. Jej atrybutem jest numer seryjny.
	
	\item \textbf{PossibleType} - enumeracja definiujące możliwe typy innych urządzeń. Są nimi samochody, telefony, sprzęt video oraz projektory.
	
	\item \textbf{License} - klasa reprezentująca byt licencji wchodząca w skład obiektu \textit{Software}.
	
	\item \textbf{Installation} - klasa asocjacyjna łącząca obiekty \textit{Software} oraz \textit{Hardware}.
	
	\item \textbf{Service} - klasa reprezentująca usługę serwisową. W skład jej atrybutów wchodzą identyfikator, opis, data wykonania oraz flaga mówiąca o tym czy usługa serwisowa była wykonana wewnątrz firmy czy została zlecona zewnętrznie.
	
	\item \textbf{Address} - klasa reprezentująca obiekt adresu. Składa się z identyfikatora, kodu pocztowego, ulicy, miasta oraz kraju. Jest agregowana przez obiekt \textit{Resource}.
	
	\item \textbf{PersonManager} - klasa zarządzająca obiektami \textit{Person} oraz \textit{Resource}. Umożliwia wykonanie operacji zapisu użytkownika zasobu jak również osoby odpowiedzialnej za zasób.
	
	\item \textbf{ResourceManager} - klasa zarządzająca obiektami \textit{Resource}, \textit{Service} oraz \textit{Address}. Umożliwia wykonywanie operacji dodania, aktualizacji, usunięcia zasobu. Dodatkowo umożliwia pobieranie zasobów wedle zadanego zapytania oraz dodanie adresu zasobu oraz usługi serwisowej danego zasobu. 
		
	\item \textbf{SoftwareManager} - klasa zarządzająca obiektami \textit{Software}. Umożliwia na dodanie informacji o instalacji oraz aktualizacji zasobu.
			
	\item \textbf{StatisticManager} - klasa korzystająca z obiektów \textit{Institution}, \textit{CompanyDepartment} oraz wykorzystująca zarządce zasobów - \textit{ResouceManager}. Umożliwia na uzyskanie statystyk zakupu zasobów w poszczególnych latach, z uwzględnieniem działu oraz instytucji. Ponadto umożliwia uzyskanie informacji o użyciu zasobu w dziale firmy, instytucji oraz liczbie wykonanych usług serwisowych w instytucji.
\end{description}

\newpage
\subsection{Diagramy sekwencji}
Przykładowe diagramy sekwencji wykorzystujące obiekty pokazane na diagramie klas.
\begin{figure}[h!]
	\centering
	\includegraphics[scale=0.4]{img/seq-pu-a}
	\caption{Edycja danych zasobu.\label{fig:seq-pu-a}}
\end{figure}
\begin{figure}[h!]
	\centering
	\includegraphics[scale=0.4]{img/seq-pu-b}
	\caption{Przypisanie osoby odpowiedzialnej za zasób.\label{fig:seq-pu-b}}
\end{figure}
\section{Logiczny model danych}

\begin{figure}[h!]
	\centering
	\includegraphics[scale=0.57, angle=270]{img/diagrams/LDM/LDM}
	\caption{Logiczny model danych\label{fig:labelLDM}}
\end{figure}
\bigskip
\subsection{Opis tabel}

\subsubsection{Address}
Odpowiada za przechowywanie adresu zasobu.
\begin{table}[H]
	\renewcommand\arraystretch{1.5}
	\renewcommand\tabcolsep{1.5pt}
\begin{tabular}{| c | c | c | c |} 
	\hline \textbf{Nazwa atrybutu} & \textbf{Znaczenie atrybutu} & \textbf{Typ danych} & \textbf{Ograniczenia} \\ 
	\hline ID & PRIMARY KEY & Numeric(10,0) & NOT NULL, UNIQUE \\ 
	\hline postCode & Kod pocztowy & Varchar(255) & REGEXP LIKE \\
	~ & ~ & ~ & \verb|[([0-9]\{2}-[0-9]\{3})| \\ 
	\hline street & Ulica & Varchar(255) &  \\ 
	\hline city & Miasto & Varchar(255) & \\ 
	\hline country & Kraj & Varchar(255) &  \\ 
	\hline resourceID & FOREIGN KEY & Numeric(10,0) & \\ 
	\hline 
\end{tabular}
\caption{Tabela Address}
\label{TAB:Address}
\end{table} 

\subsubsection{Service}
Przechowuje usługę serwisową.
\begin{table}[H]
	\renewcommand\arraystretch{1.5}
	\renewcommand\tabcolsep{1pt}
	\begin{tabular}{| c | c | c | c |} 
	\hline \textbf{Nazwa atrybutu} & \textbf{Znaczenie atrybutu} & \textbf{Typ danych} & \textbf{Ograniczenia} \\ 
	\hline ID & PRIMARY KEY & Numeric(10,0) & NOT NULL, UNIQUE \\ 
	\hline Description & Opis & Varchar(255) & \\ 
	\hline Date & Data & Timestamp &  \\ 
	\hline isExternal & Czy usługa została zlecona zewnętrznie? & Char(1) & \\ 
	\hline resourceID & FOREIGN KEY & Numeric(10,0) & \\ 
	\hline 
\end{tabular}
\caption{Tabela Service}
\label{TAB:Service}
\end{table} 

\subsubsection{Person}
Reprezentuje osobę (użytkownik, administrator).
\begin{table}[H]
	\renewcommand\arraystretch{1.5}
	\renewcommand\tabcolsep{3pt}
\begin{tabular}{| c | c | c | c |}
	\hline \textbf{Nazwa atrybutu} & \textbf{Znaczenie atrybutu} & \textbf{Typ danych} & \textbf{Ograniczenia} \\ 
	\hline ID & PRIMARY KEY & Numeric(10,0) & NOT NULL, UNIQUE \\ 
	\hline firstName & Imię & Varchar(255) & \\ 
	\hline lastName & Nazwisko & Varchar(255) &  \\ 
	\hline emailAddress & Adres email & Varchar(255) & REGEXP LIKE\\
	~ & ~ & ~ & \verb|[a-z0-9_.-]|\\ 
	~ & ~ & ~ & \verb|+@[a-z0-9_.-]|\\ 
	~ & ~ & ~ & \verb|+\.\ w {2,4}|\\ 
	\hline resourceID & FOREIGN KEY & Numeric(10,0) & \\ 
	\hline companyDepartmentID & FOREIGN KEY & Numeric(10,0) & \\ 
	\hline 
\end{tabular} 
\caption{Tabela Person}
\label{TAB:Person}
\end{table}

\subsubsection{CompanyDepartment}
Reprezentuje dział firmy.
\begin{table}[H]
	\renewcommand\arraystretch{1.5}
	\renewcommand\tabcolsep{3pt}
	\begin{tabular}{| c | c | c | c |} 
		\hline \textbf{Nazwa atrybutu} & \textbf{Znaczenie atrybutu} & \textbf{Typ danych} & \textbf{Ograniczenia} \\ 
		\hline ID & PRIMARY KEY & Numeric(10,0) & NOT NULL, UNIQUE \\ 
		\hline Name & Nazwa działu & Varchar(255) &  \\ 
		\hline institutionID & FOREIGN KEY & Numeric(10,0) & \\ 
		\hline 
	\end{tabular} 
	\caption{Tabela CompanyDepartment}
	\label{TAB:CompanyDepartment}
\end{table}

\subsubsection{CompanyDepartment}
Reprezentuje dział firmy.
\begin{table}[H]
	\renewcommand\arraystretch{1.5}
	\renewcommand\tabcolsep{3pt}
	\begin{tabular}{| c | c | c | c |} 
		\hline \textbf{Nazwa atrybutu} & \textbf{Znaczenie atrybutu} & \textbf{Typ danych} & \textbf{Ograniczenia} \\ 
		\hline ID & PRIMARY KEY & Numeric(10,0) & NOT NULL, UNIQUE \\ 
		\hline Name & Nazwa działu & Varchar(255) &  \\ 
		\hline institutionID & FOREIGN KEY & Numeric(10,0) & \\ 
		\hline 
	\end{tabular} 
	\caption{Tabela CompanyDepartment}
	\label{TAB:CompanyDepartment}
\end{table}

\subsubsection{Institution}
Reprezentuje oddział firmy.
\begin{table}[H]
	\renewcommand\arraystretch{1.5}
	\renewcommand\tabcolsep{3pt}
	\begin{tabular}{| c | c | c | c |} 
		\hline \textbf{Nazwa atrybutu} & \textbf{Znaczenie atrybutu} & \textbf{Typ danych} & \textbf{Ograniczenia} \\ 
		\hline ID & PRIMARY KEY & Numeric(10,0) & NOT NULL, UNIQUE \\ 
		\hline Name & Nazwa oddziału & Varchar(255) &  \\ 
		\hline Address & Adres & Varchar(255) & \\ 
		\hline 
	\end{tabular} 
	\caption{Tabela Institution}
	\label{TAB:Institution}
\end{table}

\subsubsection{License}
Reprezentuje licencję.
\begin{table}[H]
	\renewcommand\arraystretch{1.5}
	\renewcommand\tabcolsep{3pt}
	\begin{tabular}{| c | c | c | c |} 
		\hline \textbf{Nazwa atrybutu} & \textbf{Znaczenie atrybutu} & \textbf{Typ danych} & \textbf{Ograniczenia} \\ 
		\hline ID & PRIMARY KEY & Numeric(10,0) & NOT NULL, UNIQUE \\ 
		\hline Name & Nazwa & Varchar(255) &  \\ 
		\hline Content & Zawartość & Varchar(255) & \\ 
		\hline 
	\end{tabular} 
	\caption{Tabela License}
	\label{TAB:License}
\end{table}

\subsubsection{Installation}
Reprezentuje instalację (powiązanie software z hardware).
\begin{table}[H]
	\renewcommand\arraystretch{1.5}
	\renewcommand\tabcolsep{3pt}
	\begin{tabular}{| c | c | c | c |} 
		\hline \textbf{Nazwa atrybutu} & \textbf{Znaczenie atrybutu} & \textbf{Typ danych} & \textbf{Ograniczenia} \\ 
		\hline ID & PRIMARY KEY & Numeric(10,0) & NOT NULL, UNIQUE \\ 
		\hline Date & Data instalacji & timestamp &  \\ 
		\hline 
	\end{tabular} 
	\caption{Tabela Installation}
	\label{TAB:Installation}
\end{table}

\subsubsection{Configuration}
Reprezentuje konfigurację.
\begin{table}[H]
	\renewcommand\arraystretch{1.5}
	\renewcommand\tabcolsep{3pt}
	\begin{tabular}{| c | c | c | c |} 
		\hline \textbf{Nazwa atrybutu} & \textbf{Znaczenie atrybutu} & \textbf{Typ danych} & \textbf{Ograniczenia} \\ 
		\hline ID & PRIMARY KEY & Numeric(10,0) & NOT NULL, UNIQUE \\ 
		\hline Name & Nazwa konfigurowanej wartości & Varchar(255) &  \\ 
		\hline Name & Wartość & Varchar(255) &  \\
		\hline hardwareID & FOREIGN KEY & Numeric(10,0) & \\ 
		\hline 
	\end{tabular} 
	\caption{Tabela Configuration}
	\label{TAB:Configuration}
\end{table}

\subsubsection{Parameters}
Reprezentuje parametry (parametry i ich wartości).
\begin{table}[H]
	\renewcommand\arraystretch{1.5}
	\renewcommand\tabcolsep{3pt}
	\begin{tabular}{| c | c | c | c |} 
		\hline \textbf{Nazwa atrybutu} & \textbf{Znaczenie atrybutu} & \textbf{Typ danych} & \textbf{Ograniczenia} \\ 
		\hline ID & PRIMARY KEY & Numeric(10,0) & NOT NULL, UNIQUE \\ 
		\hline Name & Nazwa parametru & Varchar(255) &  \\ 
		\hline Name & Wartość & Varchar(255) &  \\
		\hline hardwareID & FOREIGN KEY & Numeric(10,0) & \\ 
		\hline 
	\end{tabular} 
	\caption{Tabela Parameters}
	\label{TAB:Parameters}
\end{table}
\section{Projekt interfejsu użytkownika}

W celu zwiększenia czytelności prototypów interfejsu użytkownika
zostały one zamieszczone w orientacji pionowej, co umożliwiło
zwiększenie ich rozmiaru i czcionki na nich do przystępnych rozmiarów.

\subsection{Logowanie}

\begin{figure}[H]
	\centering
        \vfill
        \noindent
        \makebox[\textwidth]{
          \includegraphics[width=0.8\textheight,angle=270]{img/screens/logowanie.png}
        }
	\caption{Ekran logowania}
\end{figure}

\subsection{Wyszukiwanie zasobu}

\begin{figure}[H]
	\centering
        \vfill
        \noindent
        \makebox[\textwidth]{
          \includegraphics[width=0.8\textheight,angle=270]{img/screens/wyszZasob.png}
        }
	\caption{Ekran wyszukiwania zasobów}
\end{figure}

\subsection{Dodawanie sprzętu komputerowego}
\begin{figure}[H]
	\centering
        \vfill
        \noindent
        \makebox[\textwidth]{
          \includegraphics[width=0.8\textheight,angle=270]{img/screens/dodawanieSprzetuKomputerowego.png}
        }
	\caption{Ekran dodawania sprzętu komputerowego}
\end{figure}

\subsection{Zapisanie informacji o użytkowniku oprogramowania}
\begin{figure}[H]
	\centering
        \vfill
        \noindent
        \makebox[\textwidth]{
          \includegraphics[width=0.8\textheight,angle=270]{img/screens/nowyUzytkownikOprogramowania.png}
        }
	\caption{Dodawanie użytkownika oprogramowania}
\end{figure}

\subsection{Rejestracja miejsca zakupu literatury lub zasobu literaturowego}
\begin{figure}[H]
	\centering
        \vfill
        \noindent
        \makebox[\textwidth]{
          \includegraphics[width=0.8\textheight,angle=270]{img/screens/miejsceZakupuLiteratura.png}
        }
	\caption{Edycja informacji na temat miejsca zakupu}
\end{figure}

\subsection{Edytowanie historii napraw sprzędu}
\begin{figure}[H]
	\centering
        \vfill
        \noindent
        \makebox[\textwidth]{
          \includegraphics[width=0.8\textheight,angle=270]{img/screens/naprawyWew.png}
        }
	\caption{Modyfikacja historii napraw sprzętu}
\end{figure}

\subsection{Prezentacja ilosci zakupionych zasobow w dzialach}
\begin{figure}[H]
	\centering
        \vfill
        \noindent
        \makebox[\textwidth]{
          \includegraphics[width=0.8\textheight,angle=270]{img/screens/raport.png}
        }
	\caption{Przykładowy raport - prezentacja ilosci zakupionych zasobow w dzialach}
\end{figure}


\end{document}

% % % % % szablon do wstawiania obrazków
%\begin{figure}[h!]
%	\centering
%	\includegraphics[scale=0.5]{sciezka-do-pliku}
%	\caption{podpis \label{fig:label1}}
%\end{figure}