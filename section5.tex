\section{Specyfikacja przypadków użycia - poziom rozszerzony}
\newcommand{\myparagraph}[1]{\paragraph{#1}\mbox{}\\}

\subsection{Aktorzy}
\begin{itemize}
\item Użytkownik systemu
\end{itemize}


\subsection{PU1 Wyszukiwanie zasobów} \label{pu1}
\myparagraph{Opis}
Przypadek wyszukiwania zasobów za pomocą zadanych kryteriów przez użytkownika.

\myparagraph{Aktorzy}
Użytkownik systemu.

\myparagraph{Warunki wstępne}
\begin{itemize}
\item Użytkownik jest zalogowany w systemie.
\end{itemize}

\myparagraph{Warunki końcowe}
Przypadek użycia nie wpływa na stan systemu.

\myparagraph{Przebieg podstawowy}
\begin{enumerate}
\item \label{pu1:f} Aktor wybiera ikonę wyszukiwania zasobów.
\item \label{pu1:s}System prezentuje okno wyszukiwania zasobów zawierające sekcję podawania kryteriów.
\item Aktor wprowadza kryteria wyszukiwania.
\item Aktor wciska przycisk ,,Wyszukaj''.
\item System prezentuje wyniki wyszukiwania.
\end{enumerate}

\myparagraph{Alternatywne przebiegi zdarzeń}
\begin{enumerate}
\item Aktor nie podał kryteriów wyszukiwania
	\begin{enumerate}[label*=\arabic*.]
	\item Kroki \ref{pu1:f} - \ref{pu1:s} jak w przebiegu podstawowym.
	\item Aktor wciska przycisk ,,Wyszukaj''.
	\item System wyświetla informację o konieczności podania kryteriów wyszukiwania.
	\end{enumerate}
\end{enumerate}

\myparagraph{Sytuacje wyjątkowe}
Brak połączenia z siecią.



\subsection{PU2 Obsługa sprzetowego}

\subsubsection{PU2.1 Dodanie nowego sprzętu komputerowego do systemu}

\myparagraph{Opis}
Przypadek dodawania nowego sprzętu komputerowego do systemu katalogowego.

\myparagraph{Aktorzy}
Użytkownik systemu.

\myparagraph{Warunki wstępne}
\begin{itemize}
\item Użytkownik jest zalogowany w systemie.
\end{itemize}

\myparagraph{Warunki końcowe}
\begin{itemize}
\item Do systemu katalogowego został dodany nowy sprzęt komputerowy.
\end{itemize}

\myparagraph{Przebieg podstawowy}
\begin{enumerate}
\item \label{pu2.1:1} Aktor wybiera ikonę dodawania sprzętu komputerowego.
\item System prezentuje okno dodawania sprzętu komputerowego.
\item Aktor podaje dane dodawanego zasobu.
\item \label{pu2.1:4} Aktor wciska przycisk ,,Dodaj''.
\item System wyświetla potwierdzenie dodania sprzętu komputerowego do systemu.
\end{enumerate}

\myparagraph{Alternatywne przebiegi zdarzeń}
\begin{enumerate}
\item Aktor nie posiada wystarczających uprawnień do dodania sprzętu komputerowego.
	\begin{enumerate}[label*=\arabic*.]
		\item Krok \ref{pu2.1:1} jak w przebiegu podstawowym.
		\item System wyświetla informację o brak wymaganych uprawnień.
	\end{enumerate}
\item Aktor podał niepoprawne lub niepełne dane sprzętu komputerowego.
	\begin{enumerate}[label*=\arabic*.]
		\item Kroki \ref{pu2.1:1} - \ref{pu2.1:4} jak w przebiegu podstawowym.
		\item System wyświetla informację o konieczności poprawienia danych.
	\end{enumerate}
\end{enumerate}

\myparagraph{Sytuacje wyjątkowe}
Brak połączenia z siecią.



\subsubsection{PU2.2 Edycja informacji o sprzęcie komputerowym}

\myparagraph{Opis}
Przypadek edycji danych sprzętu komputerowego znajdującego się w systemie katalogowym.

\myparagraph{Aktorzy}
Użytkownik systemu.

\myparagraph{Warunki wstępne}
\begin{itemize}
\item Użytkownik jest zalogowany w systemie.
\item Sprzęt komputerowy istnieje w systemie katalogowym.
\end{itemize}

\myparagraph{Warunki końcowe}
\begin{itemize}
\item Dane sprzętu komputerowego zostały zmienione.
\end{itemize}

\myparagraph{Przebieg podstawowy}
\begin{enumerate}
\item \label{pu2.2:1} Aktor wyszukuje sprzęt zgodnie z \ref{pu1}
\item \label{pu2.2:2} Aktor wybiera sprzęt do edycji oraz wciska przycisk ,,Edytuj''.
\item System prezentuje okno edycji danych sprzętu komputerowego.
\item Aktor wprowadza nowe dane.
\item \label{pu2.2:5} Aktor wciska przycisk ,,Zapisz''.
\item System wyświetla potwierdzenie zmiany danych sprzętu komputerowego.
\end{enumerate}

\myparagraph{Alternatywne przebiegi zdarzeń}
\begin{enumerate}
\item Aktor nie posiada uprawnień do edycji danych sprzętu komputerowego.
	\begin{enumerate}[label*=\arabic*.]
		\item Kroki \ref{pu2.2:1} - \ref{pu2.2:2} jak w przebiegu podstawowym.
		\item System wyświetla informację o braku uprawnień do edycji.
	\end{enumerate}
\item Aktor podał niepoprawne lub niepełne dane sprzętu komputerowego.
	\begin{enumerate}[label*=\arabic*.]
		\item Kroki \ref{pu2.2:1} - \ref{pu2.2:5} jak w przebiegu podstawowym.
		\item System wyświetla informację o konieczności poprawienia danych.
	\end{enumerate}
\end{enumerate}

\myparagraph{Sytuacje wyjątkowe}\
Brak połączenia z siecią.

\subsubsection{PU2.3 Usunięcie sprzętu komputerowego z systemu}

\myparagraph{Opis}
Przypadek użycia opisuje procedurę usunięcia sprzętu komputerowego z systemu katalogowego.

\myparagraph{Aktorzy}
Użytkownik systemu.

\myparagraph{Warunki wstępne}
\begin{itemize}
\item Użytkownik jest zalogowany w systemie.
\item Sprzęt komputerowy istnieje w systemie katalogowym.
\end{itemize}

\myparagraph{Warunki końcowe}
\begin{itemize}
\item Sprzęt komputerowy został usunięty z systemu katalogowego.
\end{itemize}

\myparagraph{Przebieg podstawowy}
\begin{enumerate}
\item \label{pu2.3:1} Aktor wyszukuje sprzęt zgodnie z \ref{pu1}
\item \label{pu2.3:2} Aktor wybiera sprzęt do usunięcia oraz wciska przycisk ,,Usuń''.
\item System wyświetla okno z prośbą o potwierdzenie wykonania operacji.
\item Aktor potwierdza wykonanie operacji.
\item System wyświetla potwierdzenie usunięcia sprzętu komputerowego z systemu.
\end{enumerate}

\myparagraph{Alternatywne przebiegi zdarzeń}
\begin{enumerate}
\item Aktor nie posiada uprawnień do usunięcia sprzętu komputerowego z systemu.
	\begin{enumerate}[label*=\arabic*.]
		\item Kroki \ref{pu2.3:1} - \ref{pu2.3:2} jak w przebiegu podstawowym.
		\item System wyświetla informację o braku uprawnień do usunięcia sprzętu komputerowego.
	\end{enumerate}
\end{enumerate}

\myparagraph{Sytuacje wyjątkowe}\
Brak połączenia z siecią.

\subsubsection{PU3.1 Dodanie nowego oprogramowania do systemu}

\myparagraph{Opis}
Przypadek użycia opisuje procedurę dodania nowego oprogramowania do systemu katalogowego.

\myparagraph{Aktorzy}
Użytkownik systemu.

\myparagraph{Warunki wstępne}
\begin{itemize}
\item Użytkownik jest zalogowany w systemie.
\end{itemize}

\myparagraph{Warunki końcowe}
\begin{itemize}
\item Oprogramowanie (wraz z danymi instalacji) zostało zapisane w systemie katalogowym.
\end{itemize}

\myparagraph{Przebieg podstawowy}
\begin{enumerate}
\item \label{pu3.1:1} Aktor wybiera ikonę dodawania oprogramowania.
\item System prezentuje okno dodawania oprogramowania.
\item Aktor podaje dane dodawanego zasobu wraz z informacjami na temat przeprowadzonych instalacji.
\item \label{pu3.1:4} Aktor wciska przycisk ,,Dodaj''.
\item System wyświetla potwierdzenie dodania oprogramowania do systemu katalogowego.
\end{enumerate}

\myparagraph{Alternatywne przebiegi zdarzeń}
\begin{enumerate}
\item Aktor nie posiada uprawnień do dodania oprogramowania.
	\begin{enumerate}[label*=\arabic*.]
		\item Krok \ref{pu3.1:1} jak w przebiegu podstawowym.
		\item System wyświetla informację o braku uprawnień do wykonania operacji dodania oprogramowania.
	\end{enumerate}
\item Aktor podał niepoprawne lub niepełne dane.
	\begin{enumerate}[label*=\arabic*.]
		\item Kroki \ref{pu3.1:1} - \ref{pu3.1:4} jak w przebiegu podstawowym.
		\item System wyświetla informację o konieczności poprawienia danych.
	\end{enumerate}
\end{enumerate}

\myparagraph{Sytuacje wyjątkowe}
Brak połączenia z siecią.

\subsubsection{PU3.2 Edycja informacji oprogramowaniu}

\myparagraph{Opis}
Przypadek użycia opisuje procedurę edycji informacji dotyczących oprogramowania znajdującego się w systemie - możliwa jest zmiana podstawowych danych oraz edycja przeprowadzonych instalacji.

\myparagraph{Aktorzy}
Użytkownik systemu.

\myparagraph{Warunki wstępne}
\begin{itemize}
\item Użytkownik jest zalogowany w systemie.
\item Oprogramowanie istnieje w systemie katalogowym.
\end{itemize}

\myparagraph{Warunki końcowe}
\begin{itemize}
\item Dane oprogramowania zostały zmienione.
\end{itemize}

\myparagraph{Przebieg podstawowy}
\begin{enumerate}
\item \label{pu3.2:1} Aktor wyszukuje oprogramowanie zgodnie z \ref{pu1}
\item \label{pu3.2:2} Aktor wybiera oprogramowanie do edycji oraz wciska przycisk ,,Edytuj''.
\item System prezentuje okno edycji danych oprogramowania.
\item Aktor wprowadza nowe dane.
\item \label{pu3.2:5} Aktor wciska przycisk ,,Zapisz''.
\item System wyświetla potwierdzenie zmiany danych oprogramowania.
\end{enumerate}

\myparagraph{Alternatywne przebiegi zdarzeń}
\begin{enumerate}
\item Aktor nie posiada uprawnień do edycji danych oprogramowania.
	\begin{enumerate}[label*=\arabic*.]
		\item Kroki \ref{pu3.2:1} - \ref{pu3.2:2} jak w przebiegu podstawowym.
		\item System wyświetla informację o braku uprawnień do edycji.
	\end{enumerate}
\item Aktor podał niepoprawne lub niepełne dane oprogramowania.
	\begin{enumerate}[label*=\arabic*.]
		\item Kroki \ref{pu3.2:1} - \ref{pu3.2:5} jak w przebiegu podstawowym.
		\item System wyświetla informację o konieczności poprawienia danych.
	\end{enumerate}
\end{enumerate}

\myparagraph{Sytuacje wyjątkowe}\
Brak połączenia z siecią.

\subsubsection{PU3.3 Usunięcie oprogramowania z systemu}

\myparagraph{Opis}
Przypadek użycia opisuje procedurę usunięcia oprogramowania (wraz z danymi na temat przeprowadzonych instalacji) z systemu katalogowego.

\myparagraph{Aktorzy}
Użytkownik systemu.

\myparagraph{Warunki wstępne}
\begin{itemize}
\item Użytkownik jest zalogowany w systemie.
\item Oprogramowanie istnieje w systemie katalogowym.
\end{itemize}

\myparagraph{Warunki końcowe}
\begin{itemize}
\item Oprogramowanie zostało usunięte z systemu katalogowego.
\end{itemize}

\myparagraph{Przebieg podstawowy}
\begin{enumerate}
\item \label{pu3.3:1} Aktor wyszukuje sprzęt zgodnie z \ref{pu1}
\item \label{pu3.3:2} Aktor wybiera oprogramowanie oraz wciska przycisk ,,Usuń''.
\item System wyświetla okno z prośbą o potwierdzenie wykonania operacji.
\item Aktor potwierdza wykonanie operacji.
\item System wyświetla potwierdzenie usunięcia oprogramowania z systemu.
\end{enumerate}

\myparagraph{Alternatywne przebiegi zdarzeń}
\begin{enumerate}
\item Aktor nie posiada uprawnień do usunięcia oprogramowania z systemu.
	\begin{enumerate}[label*=\arabic*.]
		\item Kroki \ref{pu3.3:1} - \ref{pu3.3:2} jak w przebiegu podstawowym.
		\item System wyświetla informację o braku uprawnień do usunięcia oprogramowania.
	\end{enumerate}
\end{enumerate}

\myparagraph{Sytuacje wyjątkowe}\
Brak połączenia z siecią.

\subsubsection{PU4.1 Dodanie innego sprzętu, urządzeń lub wyposażenia do systemu}

\myparagraph{Opis}
Przypadek użycia opisuje procedurę dodania sprzętu, urządzeń lub wyposażenia do systemu.

\myparagraph{Aktorzy}
Użytkownik systemu.

\myparagraph{Warunki wstępne}
\begin{itemize}
\item Użytkownik jest zalogowany w systemie.
\end{itemize}

\myparagraph{Warunki końcowe}
\begin{itemize}
\item Zasób (wraz z danymi instalacji) został zapisany w systemie katalogowym.
\end{itemize}

\myparagraph{Przebieg podstawowy}
\begin{enumerate}
\item \label{pu4.1:1} Aktor wybiera ikonę dodawania zasobu.
\item System prezentuje okno dodawania zasobu.
\item Aktor podaje dane dodawanego zasobu.
\item \label{pu4.1:4} Aktor wciska przycisk ,,Dodaj''.
\item System wyświetla potwierdzenie dodania zasobu do systemu katalogowego.
\end{enumerate}

\myparagraph{Alternatywne przebiegi zdarzeń}
\begin{enumerate}
\item Aktor nie posiada uprawnień do dodania zasobu.
	\begin{enumerate}[label*=\arabic*.]
		\item Krok \ref{pu4.1:1} jak w przebiegu podstawowym.
		\item System wyświetla informację o braku uprawnień do wykonania operacji dodania zasobu.
	\end{enumerate}
\item Aktor podał niepoprawne lub niepełne dane.
	\begin{enumerate}[label*=\arabic*.]
		\item Kroki \ref{pu4.1:1} - \ref{pu4.1:4} jak w przebiegu podstawowym.
		\item System wyświetla informację o konieczności poprawienia danych.
	\end{enumerate}
\end{enumerate}

\myparagraph{Sytuacje wyjątkowe}
Brak połączenia z siecią.

\subsubsection{PU4.2 Edycja informacji o innym sprzęcie, urządzeniu lub wyposażeniu}

\myparagraph{Opis}
Przypadek użycia opisuje procedurę edycji informacji dotyczących ogólnych zasobów.

\myparagraph{Aktorzy}
Użytkownik systemu.

\myparagraph{Warunki wstępne}
\begin{itemize}
\item Użytkownik jest zalogowany w systemie.
\item Zasób istnieje w systemie katalogowym.
\end{itemize}

\myparagraph{Warunki końcowe}
\begin{itemize}
\item Dane zasobu zostały zmienione.
\end{itemize}

\myparagraph{Przebieg podstawowy}
\begin{enumerate}
\item \label{pu4.2:1} Aktor wyszukuje zasób zgodnie z \ref{pu1}
\item \label{pu4.2:2} Aktor wybiera zasób do edycji oraz wciska przycisk ,,Edytuj''.
\item System prezentuje okno edycji danych zasobu.
\item Aktor wprowadza nowe dane.
\item \label{pu4.2:5} Aktor wciska przycisk ,,Zapisz''.
\item System wyświetla potwierdzenie zmiany danych zasobu.
\end{enumerate}

\myparagraph{Alternatywne przebiegi zdarzeń}
\begin{enumerate}
\item Aktor nie posiada uprawnień do edycji danych zasobu.
	\begin{enumerate}[label*=\arabic*.]
		\item Kroki \ref{pu4.2:1} - \ref{pu4.2:2} jak w przebiegu podstawowym.
		\item System wyświetla informację o braku uprawnień do edycji.
	\end{enumerate}
\item Aktor podał niepoprawne lub niepełne dane oprogramowania.
	\begin{enumerate}[label*=\arabic*.]
		\item Kroki \ref{pu4.2:1} - \ref{pu4.2:5} jak w przebiegu podstawowym.
		\item System wyświetla informację o konieczności poprawienia danych.
	\end{enumerate}
\end{enumerate}

\myparagraph{Sytuacje wyjątkowe}\
Brak połączenia z siecią.

\subsubsection{PU4.3 Usunięcie innego sprzętu, urządzenia lub wyposażenia z systemu}

\myparagraph{Opis}
Przypadek użycia opisuje procedurę usunięcia zasobu (ogólnego typu) z systemu katalogowego.

\myparagraph{Aktorzy}
Użytkownik systemu.

\myparagraph{Warunki wstępne}
\begin{itemize}
\item Użytkownik jest zalogowany w systemie.
\item Zasób istnieje w systemie katalogowym.
\end{itemize}

\myparagraph{Warunki końcowe}
\begin{itemize}
\item Zasób został usunięty z systemu katalogowego.
\end{itemize}

\myparagraph{Przebieg podstawowy}
\begin{enumerate}
\item \label{pu4.3:1} Aktor wyszukuje zasób zgodnie z \ref{pu1}
\item \label{pu4.3:2} Aktor wybiera zasób oraz wciska przycisk ,,Usuń''.
\item System wyświetla okno z prośbą o potwierdzenie wykonania operacji.
\item Aktor potwierdza wykonanie operacji.
\item System wyświetla potwierdzenie usunięcia zasobu z systemu.
\end{enumerate}

\myparagraph{Alternatywne przebiegi zdarzeń}
\begin{enumerate}
\item Aktor nie posiada uprawnień do usunięcia zasobu z systemu.
	\begin{enumerate}[label*=\arabic*.]
		\item Kroki \ref{pu4.3:1} - \ref{pu4.3:2} jak w przebiegu podstawowym.
		\item System wyświetla informację o braku uprawnień do usunięcia zasobu.
	\end{enumerate}
\end{enumerate}

\myparagraph{Sytuacje wyjątkowe}\
Brak połączenia z siecią.

\subsubsection{PU5.1 Dodanie nowego czasopisma lub zasobu literaturowego do systemu}

\myparagraph{Opis}
Przypadek użycia opisuje procedurę dodania czasopisma bądź zasobu literaturowego do systemu.

\myparagraph{Aktorzy}
Użytkownik systemu.

\myparagraph{Warunki wstępne}
\begin{itemize}
\item Użytkownik jest zalogowany w systemie.
\end{itemize}

\myparagraph{Warunki końcowe}
\begin{itemize}
\item Dane czasopisma lub zasobu literackiego zostały zapisane w systemie katalogowym.
\end{itemize}

\myparagraph{Przebieg podstawowy}
\begin{enumerate}
\item \label{pu5.1:1} Aktor wybiera ikonę dodawania czasopisma/zasobu literaturowego.
\item System prezentuje okno dodawania zasobu.
\item Aktor podaje dane dodawanego zasobu.
\item \label{pu5.1:4} Aktor wciska przycisk ,,Dodaj''.
\item System wyświetla potwierdzenie dodania zasobu do systemu katalogowego.
\end{enumerate}

\myparagraph{Alternatywne przebiegi zdarzeń}
\begin{enumerate}
\item Aktor nie posiada uprawnień do dodania zasobu.
	\begin{enumerate}[label*=\arabic*.]
		\item Krok \ref{pu5.1:1} jak w przebiegu podstawowym.
		\item System wyświetla informację o braku uprawnień do wykonania operacji dodania zasobu.
	\end{enumerate}
\item Aktor podał niepoprawne lub niepełne dane.
	\begin{enumerate}[label*=\arabic*.]
		\item Kroki \ref{pu5.1:1} - \ref{pu5.1:4} jak w przebiegu podstawowym.
		\item System wyświetla informację o konieczności poprawienia danych.
	\end{enumerate}
\end{enumerate}

\myparagraph{Sytuacje wyjątkowe}
Brak połączenia z siecią.

\subsubsection{PU5.2 Edycja informacji o czasopiśmie lub zasobie literaturowym}

\myparagraph{Opis}
Przypadek użycia opisuje procedurę edycji informacji dotyczących czasopisma bądź zasobu literaturowego.

\myparagraph{Aktorzy}
Użytkownik systemu.

\myparagraph{Warunki wstępne}
\begin{itemize}
\item Użytkownik jest zalogowany w systemie.
\item Czasopismo/zasób literaturowy istnieje w systemie katalogowym.
\end{itemize}

\myparagraph{Warunki końcowe}
\begin{itemize}
\item Dane zasobu zostały zmienione.
\end{itemize}

\myparagraph{Przebieg podstawowy}
\begin{enumerate}
\item \label{pu5.2:1} Aktor wyszukuje zasób zgodnie z \ref{pu1}
\item \label{pu5.2:2} Aktor wybiera zasób do edycji oraz wciska przycisk ,,Edytuj''.
\item System prezentuje okno edycji danych zasobu.
\item Aktor wprowadza nowe dane.
\item \label{pu5.2:5} Aktor wciska przycisk ,,Zapisz''.
\item System wyświetla potwierdzenie zmiany danych zasobu.
\end{enumerate}

\myparagraph{Alternatywne przebiegi zdarzeń}
\begin{enumerate}
\item Aktor nie posiada uprawnień do edycji danych zasobu.
	\begin{enumerate}[label*=\arabic*.]
		\item Kroki \ref{pu5.2:1} - \ref{pu5.2:2} jak w przebiegu podstawowym.
		\item System wyświetla informację o braku uprawnień do edycji.
	\end{enumerate}
\item Aktor podał niepoprawne lub niepełne dane oprogramowania.
	\begin{enumerate}[label*=\arabic*.]
		\item Kroki \ref{pu5.2:1} - \ref{pu5.2:5} jak w przebiegu podstawowym.
		\item System wyświetla informację o konieczności poprawienia danych.
	\end{enumerate}
\end{enumerate}

\myparagraph{Sytuacje wyjątkowe}\
Brak połączenia z siecią.

\subsubsection{PU5.3 Usunięcie czasopisma lub zasobu literaturowego z systemu}

\myparagraph{Opis}
Przypadek użycia opisuje procedurę usunięcia czasopisma/zasobu literaturowego z systemu katalogowego.

\myparagraph{Aktorzy}
Użytkownik systemu.

\myparagraph{Warunki wstępne}
\begin{itemize}
\item Użytkownik jest zalogowany w systemie.
\item Czasopismo/zasób literaturowy istnieje w systemie katalogowym.
\end{itemize}

\myparagraph{Warunki końcowe}
\begin{itemize}
\item Czasopismo/zasób literaturowy został usunięty z systemu katalogowego.
\end{itemize}

\myparagraph{Przebieg podstawowy}
\begin{enumerate}
\item \label{pu5.3:1} Aktor wyszukuje zasób zgodnie z \ref{pu1}
\item \label{pu5.3:2} Aktor wybiera zasób oraz wciska przycisk ,,Usuń''.
\item System wyświetla okno z prośbą o potwierdzenie wykonania operacji.
\item Aktor potwierdza wykonanie operacji.
\item System wyświetla potwierdzenie usunięcia zasobu z systemu.
\end{enumerate}

\myparagraph{Alternatywne przebiegi zdarzeń}
\begin{enumerate}
\item Aktor nie posiada uprawnień do usunięcia zasobu z systemu.
	\begin{enumerate}[label*=\arabic*.]
		\item Kroki \ref{pu5.3:1} - \ref{pu5.3:2} jak w przebiegu podstawowym.
		\item System wyświetla informację o braku uprawnień do usunięcia zasobu.
	\end{enumerate}
\end{enumerate}

\myparagraph{Sytuacje wyjątkowe}\
Brak połączenia z siecią.