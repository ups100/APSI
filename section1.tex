%\chapter{pierwsz}
\section{Wstęp}

% Wszelkie podobieństwo do realnych firm a tym bardziej fikcyjnych jest przypadkowe :D:D:D
Duża firma informatyczna {\bf Nabiano} ze względu na swój dynamiczny
rozwój zleciła opracowanie zintegrowanego systemu wspomagającego
zarządzania pracownikami oraz wszelkimi zasobami wykorzystywanymi w
firmie.

Ponieważ powstający system jest bardzo rozbudowany główny wykonawca
postanowił zastosować w nim architekturę modularną. W skład systemu
wchodzą następujące moduły:

\begin{itemize}
\item moduł repozytorium dokumentów wraz z mechanizmem obiegu dokumentów,
\item moduł rejestru pracowników i wykonywanych prac,
\item moduł rejestru dostępnych zasobów,
\item moduł alokacji zasobów i planowania obsady projektów,
\item moduł repozytorium wymagań dla projektów,
\item moduł repozytorium testów,
\item moduł repozytorium problemów technicznych.
\end{itemize}

Projekt oraz implementacja poszczególnych modułów została przekazana
podwykonawcom. Jednym z nich jest firma {\bf ELKA-Infor}. Jest ona
odpowiedzialna za przygotowanie projektu modułu rejestru dostępnych
zasobów oraz jego implementację.

Niniejszy dokument został przygotowany przez podwykonawce (ELKA-Infor)
jako dokumentacja projektowa do tworzonego modułu rejestru dostępnych
zasobów. Dokument ten ma na celu ułatwić współprace z głównym
wykonawcą, a także pozwolić na specyfikację wymagań wobec innych
modułów wchodzących w skład systemu.

\subsection{Opis działalności}

Firma Nabiano działa na polskim rynku od dziesięciu lat. Została ona
założona przez jej obecnego prezesa Pana Siergija Wybora. We wczesnej
fazie rozwoju firma zajmowała się sprzedarzą sprzętu komputerowego
oraz oprogramowania w kilku punktach w Białymstoku. W roku 2007 firma
rozszerzyła znacząco zakres swojej działalnosci i ukierunkowała się na
klienta biznesowego poprzez wprowadzenie do swojej oferty sprzętu
serwerowego oraz oprogramowania CMS firmy Atlasin.

Znaczący wzrost liczby zatrudnionych pracowników miał miejsce w 2009
roku, kiedy to firma postanowiła zainwestować duże środku i przy
wsparciu funduszy unijnych otworzyła dział rozwojowy w Łodzi, którego celem
było stworzenie własnego systemu CMS. Pierwsza wersja systemu powstała
już w 2010 roku i zakupiło ją kilka firm oraz jedna z największych w
Polsc korporacji.

Kolejnym krokiem milowym w rozwoju firmy było podpisanie w 2012 roku
kontraktu z rządem Korei Północnej na dostarczenie elektronicznego
systemu wyborczego na wybory parlamentarne w 2014. Projekt ten
stanowił dla firmy ogromne wyróżnienie oraz wyzwanie. Aby podołać temu
trudnemu zleceniu zwiększono zatrudnienie do ponad 1000 pracowników
oraz otwarto nowe oddziały w Warszawie oraz Pjongjang. W styczniu 2014
roku pomyślnie wdrożono wspomniany wcześniej system wyborczy. W marcu
tego samego roku odbyły się przy użyciu tego systemu wybory
parlamentarne. System działał bez żadnych zakłóceń oraz awarii. Po
wygranych przez {\it Kim Dzong Una} wyborach parlamentarnych firma
otrzymała list gratulacyjny oraz obietnicę polecenia tego systemu
wyborczego przywódcą innych państw. Już po pół roku od udanego
wdrożenia systemu wyborczego w Korei do firmy zaczęły napływać
zamówienia z cełego świata, między innymi z Białorusi, Kuby oraz
Rosji. Swoje zainteresowanie wyraziły również inne państwa
europejskie, które mają problemy ze swoimi systemamy wyborczymi.

W chwili obecnej firma Nabiano ponownie zwiększa swoje zatrudnienie i
rozpoczyn realizację kolejnych projektów. Na początku 2015 roku
planowane jest otwarcie oddziału firmy w Moskwie, a w 2016 roku w
Mińsku. Ze względu na ten dynamiczny rozwój firma postanowiła wdrożyć
system wspomagający zarządznie pracownikami, projektami oraz
wszystkimi środkami firmy.

\subsection{Przeznaczenie systemu}

