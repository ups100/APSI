%\chapter{pierwsz}
\section{Wstęp}

% Wszelkie podobieństwo do realnych firm a tym bardziej fikcyjnych jest przypadkowe :D:D:D
Duża firma informatyczna {\bf Nabiano} ze względu na swój dynamiczny
rozwój zleciła opracowanie zintegrowanego systemu wspomagającego
zarządzania pracownikami oraz wszelkimi zasobami wykorzystywanymi w
firmie.

Ponieważ powstający system jest bardzo rozbudowany główny wykonawca
postanowił zastosować w nim architekturę modularną. W skład systemu
wchodzą następujące moduły:

\begin{itemize}
\item moduł repozytorium dokumentów wraz z mechanizmem obiegu dokumentów,
\item moduł rejestru pracowników i wykonywanych prac,
\item moduł rejestru dostępnych zasobów,
\item moduł alokacji zasobów i planowania obsady projektów,
\item moduł repozytorium wymagań dla projektów,
\item moduł repozytorium testów,
\item moduł repozytorium problemów technicznych.
\end{itemize}

Projekt oraz implementacja poszczególnych modułów została przekazana
podwykonawcom. Jednym z nich jest firma {\bf ELKA-Infor}. Jest ona
odpowiedzialna za przygotowanie projektu modułu rejestru dostępnych
zasobów oraz jego implementację.

Niniejszy dokument został przygotowany przez podwykonawce (ELKA-Infor)
jako dokumentacja projektowa do tworzonego modułu rejestru dostępnych
zasobów. Dokument ten ma na celu ułatwić współprace z głównym
wykonawcą, a także pozwolić na specyfikację wymagań wobec innych
modułów wchodzących w skład systemu.

\subsection{Opis działalności}
\subsection{Przeznaczenie systemu}

