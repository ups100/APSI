\section{Analiza wymagań}

\subsection{Wymagania funkcjonalne}

\begin{myEnumerate}
\item \textbf{Zarządzanie zasobami.}
	\begin{myEnumerate}
	\item Przeszukiwanie zasobów.
	\item Katalogowanie zasobów.
	\begin{myEnumerate}
	\item Obsługa sprzętu komputerowego.
	\begin{myEnumerate}
		\item Dodanie zasobu.
		\begin{myEnumerate}
		\item Dodanie podstawowych informacji o zasobie.
		\item Dodanie informacji o elementach wchodzących w skład zasobu.
		\item Zapisanie konfiguracji sprzętu komputerowego.
		\end{myEnumerate}
		\end{myEnumerate}
		\item Edycja informacji o zasobie.
		\item Usunięcie zasobu z systemu.
	\end{myEnumerate}
	\item Obsługa oprogramowania.
	\begin{myEnumerate}
		\item Dodanie zasobu.
		\begin{myEnumerate}
			\item Dodanie podstawowych informacji o zasobie.
			\item Dodanie informacji o instalacji oprogramowania.
		\end{myEnumerate}
		\item Edycja informacji o zasobie.
		\item Usunięcie zasobu z systemu.
	\end{myEnumerate}
	\item Obsługa innego sprzętu, urządzeń i wyposażenia.
	\begin{myEnumerate}
		\item Dodanie zasobu.
		\begin{myEnumerate}
			\item Dodanie podstawowych informacji o zasobie.
		\end{myEnumerate}
		\item Edycja informacji o zasobie.
		\item Usunięcie zasobu z systemu.
	\end{myEnumerate}
	\item Obsługa czasopism oraz literatury.
	\begin{myEnumerate}
		\item Dodanie zasobu.
		\begin{myEnumerate}
			\item Dodanie podstawowych informacji o zasobie.
		\end{myEnumerate}
		\item Edycja informacji o zasobie.
		\item Usunięcie zasobu z systemu.
	\end{myEnumerate}
	\item Współpraca z modułem rejestru pracowników.
	\begin{myEnumerate}
	\item Zapisanie informacji o osobie odpowiedzialnej za dany zasób.
	\item Zapisanie informacji o użytkowniku konkretnego zasobu.
	\end{myEnumerate}
	\item Obliczanie statystyk.
	\begin{myEnumerate}
	\item Prezentacja zakupów konkretnych zasobów w poszczególnych latach.
	\item Prezentacja ilości zasobów w poszczególnych działach.
	\item Prezentacja ilość zasobów w poszczególnych placówkach.
	\end{myEnumerate}
	\end{myEnumerate}
\item \textbf{Zarządzanie serwisem sprzętu oraz oprogramowania.}
	\begin{myEnumerate}
		\item Rejestracja informacji dotyczącej miejsca zakupu.
		\item Historia napraw sprzętu.
			\begin{myEnumerate}
			\item Rejestracja informacji o wewnętrznej naprawie sprzętu.
			\item Rejestracja naprawy sprzętu przez serwis zewnętrzny.
			\end{myEnumerate}
		\item Historia obsługi oprogramowania.
		\begin{myEnumerate}
		\item Rejestracja aktualizacji oprogramowania.
		\item Rejestracja "naprawy" oprogramowania.
		\end{myEnumerate}
		\item Współpraca z repozytorium problemów.
		\begin{myEnumerate}
		\item Rejestracja operacji serwisowej.
		\end{myEnumerate}
	\end{myEnumerate}
\end{myEnumerate}
\subsection{Wymagania niefunkcjonalne}


\begin{myEnumerate}
	\item \textbf{Bezpieczeństwo}
	\begin{myEnumerate}
		\item Przechowywanie danych użytkowników zgodnie z wymogami ustawy o ochronie danych osobowych.
		\item Szyfrowanie danych wrażliwych dotyczących użytkowników zasobów.
		\item Komunikacja z systemem z wykorzystaniem SSL.
		\item Hasła użytkowników przetrzymywane w formacie bezpiecznym - w postaci wyniku funkcji skrótu hasła połączonego z losowo wygenerowaną solą.
		\item Integracja z firmowym serwerem LDAP.
	\end{myEnumerate}
	\item \textbf{Dostępność}
	\begin{myEnumerate}
		\item Dostępność systemu na poziomie 99\%.
	\end{myEnumerate}
	\item \textbf{Niezawodnosć}
	\begin{myEnumerate}
		\item Możliwość wykonywania kopii zapasowych.
		\item Możliwość zautomatyzowania procesu wykonywania kopii zapasowych.
		\item Maksymalny czas restartu systemu 2h.
		\item Możliwość odtworzenia systemu z kopii zapasowej w czasie poniżej 6h.
		\item Transakcyjny charakter operacji.
	\end{myEnumerate}
	\item \textbf{Użyteczność}
	\begin{myEnumerate}
		\item Intuicyjny interfejs graficzny.
		\item Wbudowana pomoc kontekstowa dla użytkowników systemu.
		\item Przygotowanie materiałów szkoleniowych.
		\item System dostępny w wersjach językowych: polskiej, koreańskiej.
	\end{myEnumerate}
	\item \textbf{Elastyczność}
	\begin{myEnumerate}
		\item Architektura systemu zapewnia możliwość dodawania nowych funkcji.
		\item Architektura systemu zezwala na zintegrowanie go z innymi systemami.
		\item Możliwość skalowania systemu.
	\end{myEnumerate}
	\item \textbf{Wydajność}
	\begin{myEnumerate}
		\item System przystosowany jest do równoczesnej pracy 100 pracowników.
		\item Czas średniej odpowiedzi systemu na zapytanie powinien być krótszy niż 300 ms.
	\end{myEnumerate}
\end{myEnumerate}

\subsection{Specyfikacja przypadków użycia - poziom ogólny}
\begin{longtable}{| c | c | p{.20\textwidth} | p{.40\textwidth} |} 

	\hline \textbf{ID} & \textbf{Wymaganie} & \textbf{Nazwa} & \textbf{Opis} \\ 
	\hline UC1 & 1.1. & Wyszukiwanie zasobów  & Użytkownik wyszukuje zasoby podając kryteria wyszukiwania \\ 
	\hline UC2.1 & 1.2.1.1. & Dodanie nowego sprzętu komputerowego do systemu & Użytkownik dodaje nowy sprzęt komputerowy do systemu katalogowania \\ 
	\hline UC2.2 & 1.2.2 & Edycja informacji o sprzęcie komputerowym & Użytkownik edytuje informacje dotyczące sprzętu komputerowego, który już znajduje się w systemie \\ 
	\hline UC2.3 & 1.2.3. & Usunięcie sprzętu komputerowego z systemu & Użytkownik usuwa sprzęt komputerowy z systemu \\ 
	\hline UC3.1 & 1.3.1. & Dodanie nowego oprogramowania do systemu & Użytkownik dodaje nowego oprogramowanie do systemu katalogowania \\ 
	\hline UC3.2 & 1.3.2. & Edycja informacji o oprogramowaniu & Użytkownik edytuje informacje dotyczące oprogramowania znajdującego się w systemie \\ 
	\hline UC3.3 & 1.3.3. & Usunięcie oprogramowania z systemu & Użytkownik usuwa informacje dotyczące oprogramowania z systemu  \\ 
	\hline UC4.1 & 1.4.1. & Dodanie innego sprzętu, urządzeń lub wyposażenia do systemu & Użytkownik dodaje nowy sprzęt, urządzenie lub wyposażenie do systemu \\ 
	\hline UC4.2 & 1.4.2. & Edycja informacji o innym sprzęcie, urzędzeniu lub wyposażeniu & Użytkownik edytuje informacje dotyczące innego sprzętu, urządzenia lub wyposażenia znajdującego się w systemie \\ 
	\hline UC4.3 & 1.4.3. & Usunięcie innego sprzętu, urządzenia lub wyposażenia z systemu & Użytkownik usuwa informacje dotyczące innego sprzętu, urządzenia lub wyposażenia z systemu \\
	\pagebreak
	UC5.1 & 1.5.1. & Dodanie nowego czasopisma lub zasobu literaturowego do systemu & Użytkownik dodaje nowego czasopismo lub zasób literaturowy do systemu katalogowania \\ 
	\hline UC5.2 & 1.5.2. & Edycja informacji o czasopiśmie lub zasobie literaturowym & Użytkownik edytuje informacje dotyczące czasopisma lub zasobu literaturowego znajdującego się w systemie \\ 
	\hline UC5.3 & 1.5.3.  & Usunięcie czasopisma lub zasobu literaturowego z systemu & Użytkownik usuwa informacje dotyczące czasopisma lub zasobu literaturowego z systemu \\ 
	\hline UC6.1 & 1.6.1. & Zapisanie informacji o osobie odpowiedzialnej za zasób & Użytkownik zapisuje informację o osobie odpowiedzialnej za konkretny zasób \\ 
	\hline UC6.2 & 1.6.2. & Zapisanie informacji o użytkowniku konkretnego zasobu & Użytkownik zapisuje informację o użytkowniku konkretnego zasobu \\ 
	\hline UC7.1 & 1.7.1. & Prezentacja zakupów konkretnych zasobów w poszczególnych latach & Użytkownik ma możliwość zobaczenia statystyk dot. zakupów konkretnych zasobów w poszczególnych latach \\ 
	\hline UC7.2 & 1.7.2. & Prezentacja ilości zasobów w poszczególnych działach & Użytkownik ma możliwość zobaczenia statystyk dot. zasobów w poszczególnych działach \\ 
	\hline UC7.3 & 1.7.3. & Prezentacja ilości zasobów w poszczególnych placówkach & Użytkownik ma możliwość zobaczenia statystyk dot. zakupów konkretnych zasobów w poszczególnych placówkach \\ 
	\hline UC8 & 2.1. & Rejestracja informacji o miejscu zakupu zasobu & Użytkownik rejestruje miejsce zakupu konkretnego zasobu \\ 
	\hline UC9.1 & 2.2.1. & Rejestracja informacji o wewnętrznej naprawie sprzętu & Użytkownik rejestruje fakt przeprowadzenia wewnętrznej naprawy sprzętu \\ 
	\hline UC9.2 & 2.2.2. & Rejestracja informacji o zewnętrznej naprawie sprzętu & Użytkownik rejestruje fakt przeprowadzenia zewnętrznej naprawy sprzętu \\ 
	\hline UC10 & 2.3.1. & Rejestracja aktualizacji oprogramowania & Użytkownik rejestruje przeprowadzenie aktualizacji oprogramowania \\ 
	\hline UC11 & 2.4.1. & Rejestracja operacji serwisowej & Użytkownik rejestruje przeprowadzenie operacji serwisowej zasobu \\ 
	\hline 
\end{longtable} 

