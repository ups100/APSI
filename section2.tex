\section{Analiza wymagań}

\subsection{Wymagania funkcjonalne}

\begin{myEnumerate}
\item \textbf{Zarządzanie zasobami.}
	\begin{myEnumerate}
	\item Przeszukiwanie zasobów.
	\item Katalogowanie zasobów.
	\begin{myEnumerate}
	\item Obsługa sprzętu komputerowego.
	\begin{myEnumerate}
		\item Dodanie zasobu.
		\begin{myEnumerate}
		\item Dodanie podstawowych informacji o zasobie.
		\item Dodanie informacji o elementach wchodzących w skład zasobu.
		\item Zapisanie konfiguracji sprzętu komputerowego.
		\end{myEnumerate}
		\end{myEnumerate}
		\item Edycja informacji o zasobie.
		\item Usunięcie zasobu z systemu.
	\end{myEnumerate}
	\item Obsługa oprogramowania.
	\begin{myEnumerate}
		\item Dodanie zasobu.
		\begin{myEnumerate}
			\item Dodanie podstawowych informacji o zasobie.
			\item Dodanie informacji o instalacji oprogramowania.
		\end{myEnumerate}
		\item Edycja informacji o zasobie.
		\item Usunięcie zasobu z systemu.
	\end{myEnumerate}
	\item Obsługa innego sprzętu, urządzeń i wyposażenia.
	\begin{myEnumerate}
		\item Dodanie zasobu.
		\begin{myEnumerate}
			\item Dodanie podstawowych informacji o zasobie.
		\end{myEnumerate}
		\item Edycja informacji o zasobie.
		\item Usunięcie zasobu z systemu.
	\end{myEnumerate}
	\item Obsługa czasopism oraz literatury.
	\begin{myEnumerate}
		\item Dodanie zasobu.
		\begin{myEnumerate}
			\item Dodanie podstawowych informacji o zasobie.
		\end{myEnumerate}
		\item Edycja informacji o zasobie.
		\item Usunięcie zasobu z systemu.
	\end{myEnumerate}
	\item Współpraca z modułem rejestru pracowników.
	\begin{myEnumerate}
	\item Zapisanie informacji o osobie odpowiedzialnej za dany zasób.
	\item Zapisanie informacji o użytkowniku konkretnego zasobu.
	\end{myEnumerate}
	\item Obliczanie statystyk.
	\begin{myEnumerate}
	\item Prezentacja zakupów konkretnych zasobów w poszczególnych latach.
	\item Prezentacja ilości zasobów w poszczególnych działach.
	\item Prezentacja ilość zasobów w poszczególnych placówkach.
	\end{myEnumerate}
	\end{myEnumerate}
\item \textbf{Zarządzanie serwisem sprzętu oraz oprogramowania.}
	\begin{myEnumerate}
		\item Rejestracja informacji dotyczącej miejsca zakupu.
		\item Historia napraw sprzętu.
			\begin{myEnumerate}
			\item Rejestracja informacji o wewnętrznej naprawie sprzętu.
			\item Rejestracja naprawy sprzętu przez serwis zewnętrzny.
			\end{myEnumerate}
		\item Historia obsługi oprogramowania.
		\begin{myEnumerate}
		\item Rejestracja aktualizacji oprogramowania.
		\item Rejestracja "naprawy" oprogramowania.
		\end{myEnumerate}
		\item Współpraca z repozytorium problemów.
		\begin{myEnumerate}
		\item Rejestracja operacji serwisowej.
		\end{myEnumerate}
	\end{myEnumerate}
\end{myEnumerate}
\subsection{Wymagania niefunkcjonalne}


\begin{myEnumerate}
	\item \textbf{Bezpieczeństwo}
	\begin{myEnumerate}
		\item Przechowywanie danych użytkowników zgodnie z wymogami ustawy o ochronie danych osobowych.
		\item Szyfrowanie danych wrażliwych dotyczących użytkowników zasobów.
		\item Komunikacja z systemem z wykorzystaniem SSL.
		\item Hasła użytkowników przetrzymywane w formacie bezpiecznym - w postaci wyniku funkcji skrótu hasła połączonego z losowo wygenerowaną solą.
		\item Integracja z firmowym serwerem LDAP.
	\end{myEnumerate}
	\item \textbf{Dostępność}
	\begin{myEnumerate}
		\item Dostępność systemu na poziomie 99\%.
	\end{myEnumerate}
	\item \textbf{Niezawodnosć}
	\begin{myEnumerate}
		\item Możliwość wykonywania kopii zapasowych.
		\item Możliwość zautomatyzowania procesu wykonywania kopii zapasowych.
		\item Maksymalny czas restartu systemu 2h.
		\item Możliwość odtworzenia systemu z kopii zapasowej w czasie poniżej 6h.
		\item Transakcyjny charakter operacji.
	\end{myEnumerate}
	\item \textbf{Użyteczność}
	\begin{myEnumerate}
		\item Intuicyjny interfejs graficzny.
		\item Wbudowana pomoc kontekstowa dla użytkowników systemu.
		\item Przygotowanie materiałów szkoleniowych.
		\item System dostępny w wersjach językowych: polskiej, koreańskiej.
	\end{myEnumerate}
	\item \textbf{Elastyczność}
	\begin{myEnumerate}
		\item Architektura systemu zapewnia możliwość dodawania nowych funkcji.
		\item Architektura systemu zezwala na zintegrowanie go z innymi systemami.
		\item Możliwość skalowania systemu.
	\end{myEnumerate}
	\item \textbf{Wydajność}
	\begin{myEnumerate}
		\item System przystosowany jest do równoczesnej pracy 100 pracowników.
		\item Czas średniej odpowiedzi systemu na zapytanie powinien być krótszy niż 300 ms.
	\end{myEnumerate}
\end{myEnumerate}

\subsection{Specyfikacja przypadków użycia - poziom ogólny}
