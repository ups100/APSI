\section{Analiza wymagań}

\subsection{Wymagania funkcjonalne}
%					\\ \rule{\linewidth}{.1pt}
Lista wymagań funkcjonalnych wraz z priorytetami gdzie 1 oznacza najwyższy priorytet, natomiast 3 najniższy. \\

\hfill Priorytet
\begin{myEnumerate}
\item\label{f1} \textbf{Zarządzanie zasobami.}
\hfill 1
	\begin{myEnumerate}
	\item\label{f2} Przeszukiwanie zasobów.
	\hfill 1
	\item\label{f3} Katalogowanie zasobów.
	\hfill 1
	\begin{myEnumerate}

	\item\label{f4} Obsługa sprzętu komputerowego.
	\hfill 1
	\begin{myEnumerate}
		\item\label{f5} Dodanie zasobu.
		\hfill 1
		\begin{myEnumerate}
		\item\label{f6} Dodanie podstawowych informacji o zasobie.
		 \hfill 1
		\item\label{f7} Dodanie informacji o elementach wchodzących w skład \\  zasobu.
		\hfill 1
		\item\label{f8} Zapisanie konfiguracji sprzętu komputerowego.
		\hfill 2
		\end{myEnumerate}
		\item\label{f9} Edycja informacji o zasobie.
		\hfill 2
		\item\label{f10} Usunięcie zasobu z systemu.
		\hfill 3
	\end{myEnumerate}

	\item\label{f11} Obsługa oprogramowania.
	\hfill 1

	\begin{myEnumerate}
		\item\label{f12} Dodanie zasobu.
		\hfill 1
		\begin{myEnumerate}
			\item\label{f13} Dodanie podstawowych informacji o zasobie.
			\hfill 1
			\item\label{f14} Dodanie informacji o instalacji oprogramowania.
			\hfill 2
		\end{myEnumerate}
		\item\label{f15} Edycja informacji o zasobie.
		\hfill 2
		\item\label{f16} Usunięcie zasobu z systemu.
		\hfill 3
	\end{myEnumerate}

	\item\label{f17} Obsługa innego sprzętu, urządzeń i wyposażenia.
	\hfill 1
	\begin{myEnumerate}
		\item\label{f18} Dodanie zasobu.
		\hfill 1
		\begin{myEnumerate}
			\item\label{f19} Dodanie podstawowych informacji o zasobie.
			\hfill 1
		\end{myEnumerate}
		\item\label{f20} Edycja informacji o zasobie.
		\hfill 2
		\item\label{f21} Usunięcie zasobu z systemu.
		\hfill 3
	\end{myEnumerate}

	\item\label{f22} Obsługa czasopism oraz literatury.
	\hfill 1
	\begin{myEnumerate}
		\item\label{f23} Dodanie zasobu.
		\hfill 1
		\begin{myEnumerate}
			\item\label{f24} Dodanie podstawowych informacji o zasobie.
			\hfill 1
		\end{myEnumerate}
		\item\label{f25} Edycja informacji o zasobie.
		\hfill 2
		\item\label{f26} Usunięcie zasobu z systemu.
		\hfill 3
	\end{myEnumerate}

	\end{myEnumerate}

	\item\label{f27} Współpraca z modułem rejestru pracowników.
	\hfill 1
	\begin{myEnumerate}
	\item\label{f28} Zapisanie informacji o osobie odpowiedzialnej za dany zasób.
	\hfill 1
	\item\label{f29} Zapisanie informacji o użytkowniku konkretnego
	 zasobu.
	 \hfill 1
	\end{myEnumerate}

	\item\label{f30} Obliczanie statystyk.
	\hfill 3
	\begin{myEnumerate}
	\item\label{f31} Prezentacja zakupów konkretnych zasobów w poszczególnych \\ latach.
	\hfill 3
	\item\label{f32} Prezentacja ilości zasobów w poszczególnych działach.
	\hfill 3
	\item\label{f33} Prezentacja ilość zasobów w poszczególnych placówkach.
	\hfill 3
	\end{myEnumerate}
	\end{myEnumerate}


\item\label{f34} \textbf{Zarządzanie serwisem sprzętu oraz oprogramowania.}
	\hfill 1
	\begin{myEnumerate}
		\item\label{f35} Rejestracja informacji dotyczącej miejsca zakupu.
			\hfill 1
		\item\label{f36} Historia napraw sprzętu.
		\hfill 1
			\begin{myEnumerate}
			\item\label{f37} Rejestracja informacji o wewnętrznej naprawie sprzętu.
			\hfill 1
			\item\label{f38} Rejestracja naprawy sprzętu przez serwis zewnętrzny.
			\hfill 1
			\end{myEnumerate}
		\item\label{f39} Historia obsługi oprogramowania.
		\hfill 1
		\begin{myEnumerate}
		\item\label{f40} Rejestracja aktualizacji oprogramowania.
			\hfill 1
		\item\label{f41} Rejestracja "naprawy" oprogramowania.
					\hfill 1
		\end{myEnumerate}
		\item\label{f42} Współpraca z repozytorium problemów.
					\hfill 1
		\begin{myEnumerate}
		\item\label{f43} Rejestracja operacji serwisowej.
					\hfill 1
		\end{myEnumerate}
	\end{myEnumerate}
\end{myEnumerate}

% ostatnia etykieta labelki: f43

\subsection{Wymagania niefunkcjonalne}
\label{wymagania_niefunkcjonalne}

\begin{myEnumerate}
	\item \textbf{Bezpieczeństwo}
          \hfill 1
	\begin{myEnumerate}
		\item Przechowywanie danych użytkowników zgodnie z wymogami ustawy o ochronie danych osobowych.
                  \hfill 1
		\item Szyfrowanie danych wrażliwych dotyczących użytkowników zasobów.
                  \hfill 1
		\item Komunikacja z systemem z wykorzystaniem SSL.
                  \hfill 1
		\item Hasła użytkowników przetrzymywane w formacie bezpiecznym - w postaci wyniku funkcji skrótu hasła połączonego z losowo wygenerowaną solą.
                  \hfill 1
		\item Integracja z firmowym serwerem LDAP.
                  \hfill 1
	\end{myEnumerate}
	\item \textbf{Dostępność}
          \hfill 2
	\begin{myEnumerate}
		\item Dostępność systemu na poziomie 99\%.
                  \hfill 1
                \item Dostępność systemu na poziomie 99,9\%.
                  \hfill 3
	\end{myEnumerate}
	\item \textbf{Niezawodność}
          \hfill 1
	\begin{myEnumerate}
		\item Możliwość wykonywania kopii zapasowych.
                  \hfill 1
		\item Możliwość zautomatyzowania procesu wykonywania kopii zapasowych.
                  \hfill 2
		\item Maksymalny czas restartu systemu 2h.
                  \hfill 1
		\item Możliwość odtworzenia systemu z kopii zapasowej w czasie poniżej 6h.
                  \hfill 2
		\item Transakcyjny charakter operacji.
                  \hfill 1
	\end{myEnumerate}
	\item \textbf{Użyteczność}
          \hfill 1
	\begin{myEnumerate}
		\item Intuicyjny interfejs graficzny.
                  \hfill 1
		\item Wbudowana pomoc kontekstowa dla użytkowników systemu.
                  \hfill 1
		\item Przygotowanie materiałów szkoleniowych.
                  \hfill 1
		\item System dostępny w angielskiej wersji językowej.
                  \hfill 1
                \item System dostępny w polskiej wersji językowej.
                  \hfill 2
                \item System dostępny w koreańskiej wersji językowej.
                  \hfill 1
	\end{myEnumerate}
	\item \textbf{Elastyczność}
                  \hfill 1
	\begin{myEnumerate}
		\item Architektura systemu zapewnia możliwość dodawania nowych funkcji.
                  \hfill 1
		\item Architektura systemu zezwala na zintegrowanie go z innymi systemami.
                  \hfill 1
		\item Możliwość skalowania systemu.
                  \hfill 2
	\end{myEnumerate}
	\item \textbf{Wydajność}
                  \hfill 2
	\begin{myEnumerate}
		\item System przystosowany jest do równoczesnej pracy 300 pracowników.
                  \hfill 2
		\item System przystosowany jest do zarządzania co najmniej 30000 przedmiotów i licencji.
                  \hfill 2
		\item Czas średniej odpowiedzi systemu na zapytanie powinien być krótszy niż 300 ms.
                  \hfill 3
	\end{myEnumerate}
\end{myEnumerate}

\subsection{Specyfikacja przypadków użycia - poziom ogólny}

\vspace{.03\textheight}
\begin{center}
  {\Large\bf UC1 Wyszukiwanie zasobów} \\
\end{center}

\vspace{.02\textheight}

\begin{tabular}{p{.2\textwidth}p{0.7\textwidth}}
\hfill {\bf ID} & UC1.1 \\
\hfill {\bf Nazwa} & Wyszukiwanie zasobów \\
\hfill {\bf Wymagania} & 1.1 \\
\hfill {\bf Opis} & Użytkownik wyszukuje zasoby podając kryteria wyszukiwania \\
\end{tabular}

\vspace{.03\textheight}
\begin{center}
  {\Large\bf UC2 Obsługa sprzętu komputerowego} \\
\end{center}
\vspace{.02\textheight}

\begin{tabular}{p{.2\textwidth}p{0.7\textwidth}}
\hfill {\bf ID} & UC1.1 \\
\hfill {\bf Nazwa} & Dodanie nowego sprzętu komputerowego do systemu \\
\hfill {\bf Wymagania} & 1.2.1.1\\
\hfill {\bf Opis} & Użytkownik dodaje nowy sprzęt komputerowy do systemu katalogowania, podając podstawowe informacje o sprzęcie, skład elementów wchodzących w jego skład oraz zapisuje konfiguracje sprzętu \\
\end{tabular}

\vspace{.05\textheight}

\begin{tabular}{p{.2\textwidth}p{0.7\textwidth}}
\hfill {\bf ID} & UC2.2 \\
\hfill {\bf Nazwa} & Edycja informacji o sprzęcie komputerowym \\
\hfill {\bf Wymagania} & 1.2.2 \\
\hfill {\bf Opis} & Użytkownik edytuje informacje dotyczące sprzętu komputerowego, który już znajduje się w systemie - może zmienić podstawowe informacje o sprzęcie, skład elementów wchodzących w jego skład oraz zmienić konfiguracje sprzętu \\
\end{tabular}

\vspace{.05\textheight}

\begin{tabular}{p{.2\textwidth}p{0.7\textwidth}}
\hfill {\bf ID} & UC2.3 \\
\hfill {\bf Nazwa} & Usunięcie sprzętu komputerowego z systemu \\
\hfill {\bf Wymagania} & 1.2.3 \\
\hfill {\bf Opis} & Użytkownik usuwa sprzęt komputerowy z systemu \\
\end{tabular}

\vspace{.03\textheight}
\begin{center}
  {\Large\bf UC3 Obsługa oprogramowania} \\
\end{center}
\vspace{.02\textheight}


\begin{tabular}{p{.2\textwidth}p{0.7\textwidth}}
\hfill {\bf ID} & UC3.1 \\
\hfill {\bf Nazwa} & Dodanie nowego oprogramowania do systemu \\
\hfill {\bf Wymagania} & 1.3.1 \\
\hfill {\bf Opis} & Użytkownik dodaje nowego oprogramowanie do systemu katalogowania, podając podstawowe informacje o oprogramowaniu oraz zapisując informację na temat przeprowadzonych instalacji \\
\end{tabular}

\vspace{.05\textheight}

\begin{tabular}{p{.2\textwidth}p{0.7\textwidth}}
\hfill {\bf ID} & UC3.2 \\
\hfill {\bf Nazwa} &  Edycja informacji o oprogramowaniu \\
\hfill {\bf Wymagania} & 1.3.2 \\
\hfill {\bf Opis} & Użytkownik edytuje informacje dotyczące oprogramowania znajdującego się w systemie - może zmienić podstawowe informacje o oprogramowaniu oraz edytować informację na temat przeprowadzonych instalacji \\
\end{tabular}

\vspace{.05\textheight}

\begin{tabular}{p{.2\textwidth}p{0.7\textwidth}}
\hfill {\bf ID} & UC3.3 \\
\hfill {\bf Nazwa} & Usunięcie oprogramowania z systemu \\
\hfill {\bf Wymagania} & 1.3.3 \\
\hfill {\bf Opis} & Użytkownik usuwa informacje dotyczące oprogramowania z systemu \\
\end{tabular}

\vspace{.03\textheight}
\begin{center}
  {\Large\bf UC4 Obsługa innego sprzętu, urządzeń i wyposażenia } \\
\end{center}
\vspace{.02\textheight}

\begin{tabular}{p{.2\textwidth}p{0.7\textwidth}}
\hfill {\bf ID} & UC4.1  \\
\hfill {\bf Nazwa} & Dodanie innego sprzętu, urządzeń lub wyposażenia do systemu \\
\hfill {\bf Wymagania} & 1.4.1 \\
\hfill {\bf Opis} &  Użytkownik dodaje nowy sprzęt, urządzenie lub wyposażenie do systemu \\
\end{tabular}

\vspace{.05\textheight}

\begin{tabular}{p{.2\textwidth}p{0.7\textwidth}}
\hfill {\bf ID} & UC4.2 \\
\hfill {\bf Nazwa} & Edycja informacji o innym sprzęcie, urządzeniu lub wyposażeniu \\
\hfill {\bf Wymagania} & 1.4.2  \\
\hfill {\bf Opis} & Użytkownik edytuje informacje dotyczące innego sprzętu, urządzenia lub wyposażenia znajdującego się w systemie \\
\end{tabular}

\vspace{.05\textheight}

\begin{tabular}{p{.2\textwidth}p{0.7\textwidth}}
\hfill {\bf ID} & UC4.3  \\
\hfill {\bf Nazwa} & Usunięcie innego sprzętu, urządzenia lub wyposażenia z systemu \\
\hfill {\bf Wymagania} & 1.4.3 \\
\hfill {\bf Opis} & Użytkownik usuwa informacje dotyczące innego sprzętu, urządzenia lub wyposażenia z systemu \\
\end{tabular}

\vspace{.03\textheight}
\begin{center}
  {\Large\bf UC5 Obsługa czasopism oraz literatur } \\
\end{center}
\vspace{.02\textheight}

\begin{tabular}{p{.2\textwidth}p{0.7\textwidth}}
\hfill {\bf ID} & UC5.1 \\
\hfill {\bf Nazwa} & Dodanie nowego czasopisma lub zasobu literaturowego do systemu \\
\hfill {\bf Wymagania} & 1.5.1 \\
\hfill {\bf Opis} & Użytkownik dodaje nowego czasopismo lub zasób literaturowy do systemu katalogowania \\
\end{tabular}

\vspace{.05\textheight}

\begin{tabular}{p{.2\textwidth}p{0.7\textwidth}}
\hfill {\bf ID} & UC5.2 \\
\hfill {\bf Nazwa} & Edycja informacji o czasopiśmie lub zasobie literaturowym  \\
\hfill {\bf Wymagania} & 1.5.2 \\
\hfill {\bf Opis} &  Użytkownik edytuje informacje dotyczące czasopisma lub zasobu literaturowego znajdującego się w systemie \\
\end{tabular}

\vspace{.05\textheight}

\begin{tabular}{p{.2\textwidth}p{0.7\textwidth}}
\hfill {\bf ID} & UC5.3 \\
\hfill {\bf Nazwa} & Usunięcie czasopisma lub zasobu literaturowego z systemu \\
\hfill {\bf Wymagania} & 1.5.3 \\
\hfill {\bf Opis} & Użytkownik usuwa informacje dotyczące czasopisma lub zasobu literaturowego z systemu \\
\end{tabular}

\vspace{.03\textheight}
\begin{center}
  {\Large\bf UC6 Współpraca z modułem rejestru pracowników}
\end{center}
\vspace{.02\textheight}

\begin{tabular}{p{.2\textwidth}p{0.7\textwidth}}
\hfill {\bf ID} & UC6.1 \\
\hfill {\bf Nazwa} &  Zapisanie informacji o osobie odpowiedzialnej za zasób \\
\hfill {\bf Wymagania} & 1.6.1 \\
\hfill {\bf Opis} & Użytkownik zapisuje informację o osobie odpowiedzialnej za konkretny zasób, wybierając ją z listy pracowników - za zasób odpowiedzialna jest dokładnie jeden pracownik \\
\end{tabular}

\vspace{.05\textheight}

\begin{tabular}{p{.2\textwidth}p{0.7\textwidth}}
\hfill {\bf ID} & UC6.2 \\
\hfill {\bf Nazwa} & Zapisanie informacji o użytkowniku konkretnego zasobu \\
\hfill {\bf Wymagania} & 1.6.2 \\
\hfill {\bf Opis} &  Użytkownik zapisuje informację o użytkowniku konkretnego zasobu,  wybierając go z listy pracowników - dany zasób może nie być użytkowany przez żadną osobę lub może być użytkowany przez dokładnie jedną lub kilku użytkowników (w zależności od rodzaju zasobu) \\
\end{tabular}

\vspace{.03\textheight}
\begin{center}
  {\Large\bf UC7 Obliczanie statystyk }
\end{center}
\vspace{.02\textheight}

\begin{tabular}{p{.2\textwidth}p{0.7\textwidth}}
\hfill {\bf ID} & UC7.1 \\
\hfill {\bf Nazwa} &  Prezentacja zakupów konkretnych zasobów w poszczególnych latach \\
\hfill {\bf Wymagania} & 1.7.1 \\
\hfill {\bf Opis} & Użytkownik ma możliwość zobaczenia statystyk dot. zakupów konkretnych zasobów w poszczególnych latach (podział możliwy zarówno na całą kategorie zasobu jak i jeden lub kilka zasobów z danej kategorii) \\
\end{tabular}

\vspace{.05\textheight}

\begin{tabular}{p{.2\textwidth}p{0.7\textwidth}}
\hfill {\bf ID} &  UC7.2.1 \\
\hfill {\bf Nazwa} & Prezentacja ilości zakupionych zasobów w poszczególnych działach \\
\hfill {\bf Wymagania} & 1.7.2 \\
\hfill {\bf Opis} & Użytkownik ma możliwość zobaczenia statystyk dot. zakupionych zasobów w poszczególnych działach (podział możliwy zarówno na całą kategorie zasobu jak i jeden lub kilka zasobów z danej kategorii) \\
\end{tabular}

\vspace{.05\textheight}

\begin{tabular}{p{.2\textwidth}p{0.7\textwidth}}
\hfill {\bf ID} & UC7.2.2 \\
\hfill {\bf Nazwa} & Prezentacja ilości używanych zasobów w poszczególnych działach \\
\hfill {\bf Wymagania} & 1.7.3 \\
\hfill {\bf Opis} & Użytkownik ma możliwość zobaczenia statystyk dot. używanych zasobów w poszczególnych działach (podział możliwy zarówno na całą kategorie zasobu jak i jeden lub kilka zasobów z danej kategorii) \\
\end{tabular}

\vspace{.05\textheight}

\begin{tabular}{p{.2\textwidth}p{0.7\textwidth}}
\hfill {\bf ID} & UC7.2.3 \\
\hfill {\bf Nazwa} &  Prezentacja ilości napraw zasobów w poszczególnych działach \\
\hfill {\bf Wymagania} & 1.7.4 \\
\hfill {\bf Opis} &  Użytkownik ma możliwość zobaczenia statystyk dot. napraw zasobów w poszczególnych działach (podział możliwy zarówno na całą kategorie zasobu jak i jeden lub kilka zasobów z danej kategorii) \\
\end{tabular}

\vspace{.05\textheight}

\begin{tabular}{p{.2\textwidth}p{0.7\textwidth}}
\hfill {\bf ID} & UC7.3.1 \\
\hfill {\bf Nazwa} & Prezentacja ilości zakupionych zasobów w poszczególnych placówkach \\
\hfill {\bf Wymagania} & 1.7.5 \\
\hfill {\bf Opis} & Użytkownik ma możliwość zobaczenia statystyk dot. zakupionych zasobów w poszczególnych placówkach (podział możliwy zarówno na całą kategorie zasobu jak i jeden lub kilka zasobów z danej kategorii) \\
\end{tabular}

\vspace{.05\textheight}

\begin{tabular}{p{.2\textwidth}p{0.7\textwidth}}
\hfill {\bf ID} & UC7.3.2 \\
\hfill {\bf Nazwa} &  Prezentacja ilości używanych zasobów w poszczególnych placówkach \\
\hfill {\bf Wymagania} & 1.7.6  \\
\hfill {\bf Opis} &  Użytkownik ma możliwość zobaczenia statystyk dot. używanych zasobów w poszczególnych placówkach (podział możliwy zarówno na całą kategorie zasobu jak i jeden lub kilka zasobów z danej kategorii) \\
\end{tabular}

\vspace{.05\textheight}

\begin{tabular}{p{.2\textwidth}p{0.7\textwidth}}
\hfill {\bf ID} & UC7.3.3 \\
\hfill {\bf Nazwa} & Prezentacja ilości napraw zasobów w poszczególnych placówkach \\
\hfill {\bf Wymagania} & 1.7.7 \\
\hfill {\bf Opis} &  Użytkownik ma możliwość zobaczenia statystyk dot. napraw zasobów w poszczególnych placówkach (podział możliwy zarówno na całą kategorie zasobu jak i jeden lub kilka zasobów z danej kategorii) \\
\end{tabular}

\vspace{.03\textheight}
\begin{center}
  {\Large\bf UC8 Rejestracja informacji dotyczącej miejsca zakupu} \\
\end{center}
\vspace{.02\textheight}


\begin{tabular}{p{.2\textwidth}p{0.7\textwidth}}
\hfill {\bf ID} & UC8.1 \\
\hfill {\bf Nazwa} & Rejestracja informacji o miejscu zakupu zasobu  \\
\hfill {\bf Wymagania} &  2.1 \\
\hfill {\bf Opis} &  Użytkownik rejestruje miejsce zakupu konkretnego zasobu \\
\end{tabular}

\vspace{.03\textheight}
\begin{center}
  {\Large\bf UC9 Historia napraw sprzętu} \\
\end{center}
\vspace{.02\textheight}

\begin{tabular}{p{.2\textwidth}p{0.7\textwidth}}
\hfill {\bf ID} & UC9.1 \\
\hfill {\bf Nazwa} & Rejestracja informacji o wewnętrznej naprawie sprzętu \\
\hfill {\bf Wymagania} &  2.2.1 \\
\hfill {\bf Opis} & Użytkownik rejestruje fakt przeprowadzenia wewnętrznej naprawy sprzętu, wybierając pracownika, który przeprowadził naprawę z listy pracowników \\
\end{tabular}

\vspace{.05\textheight}

\begin{tabular}{p{.2\textwidth}p{0.7\textwidth}}
\hfill {\bf ID} & UC9.2 \\
\hfill {\bf Nazwa} &  Rejestracja informacji o zewnętrznej naprawie sprzętu \\
\hfill {\bf Wymagania} &  2.2.2 \\
\hfill {\bf Opis} &  Użytkownik rejestruje fakt przeprowadzenia zewnętrznej naprawy sprzętu \\
\end{tabular}

\vspace{.03\textheight}
\begin{center}
  {\Large\bf UC10 Rejestracja aktualizacji oprogramowania} \\
\end{center}
\vspace{.02\textheight}

\begin{tabular}{p{.2\textwidth}p{0.7\textwidth}}
\hfill {\bf ID} & UC10.1 \\
\hfill {\bf Nazwa} & Rejestracja aktualizacji oprogramowania \\
\hfill {\bf Wymagania} & 2.3.1 \\
\hfill {\bf Opis} & Użytkownik rejestruje przeprowadzenie aktualizacji oprogramowania \\
\end{tabular}

\vspace{.03\textheight}
\begin{center}
  {\Large\bf UC11 Rejestracja operacji serwisowej} \\
\end{center}
\vspace{.02\textheight}

\begin{tabular}{p{.2\textwidth}p{0.7\textwidth}}
\hfill {\bf ID} & UC11.1 \\
\hfill {\bf Nazwa} &  Rejestracja operacji serwisowej \\
\hfill {\bf Wymagania} &  2.4.1 \\
\hfill {\bf Opis} & Użytkownik rejestruje przeprowadzenie operacji serwisowej zasobu  \\
\end{tabular}

