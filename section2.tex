\section{Analiza wymagań}

\subsection{Wymagania funkcjonalne}
%					\\ \rule{\linewidth}{.1pt}
Lista wymagań funkcjonalnych wraz z priorytetami gdzie 1 oznacza najwyższy priorytet, natomiast 3 najniższy. \\

\hfill Priorytet
\begin{myEnumerate}
\item\label{f1} \textbf{Zarządzanie zasobami.} 
\hfill 1
	\begin{myEnumerate}
	\item\label{f2} Przeszukiwanie zasobów. 
	\hfill 1
	\item\label{f3} Katalogowanie zasobów. 
	\hfill 1
	\begin{myEnumerate}

	\item\label{f4} Obsługa sprzętu komputerowego. 
	\hfill 1
	\begin{myEnumerate}
		\item\label{f5} Dodanie zasobu. 
		\hfill 1
		\begin{myEnumerate}
		\item\label{f6} Dodanie podstawowych informacji o zasobie.
		 \hfill 1
		\item\label{f7} Dodanie informacji o elementach wchodzących w skład \\  zasobu. 
		\hfill 1
		\item\label{f8} Zapisanie konfiguracji sprzętu komputerowego. 
		\hfill 2
		\end{myEnumerate}
		\item\label{f9} Edycja informacji o zasobie. 
		\hfill 2
		\item\label{f10} Usunięcie zasobu z systemu. 
		\hfill 3
	\end{myEnumerate}

	\item\label{f11} Obsługa oprogramowania. 
	\hfill 1

	\begin{myEnumerate}
		\item\label{f12} Dodanie zasobu. 
		\hfill 1
		\begin{myEnumerate}
			\item\label{f13} Dodanie podstawowych informacji o zasobie. 
			\hfill 1
			\item\label{f14} Dodanie informacji o instalacji oprogramowania. 
			\hfill 2
		\end{myEnumerate}
		\item\label{f15} Edycja informacji o zasobie. 
		\hfill 2
		\item\label{f16} Usunięcie zasobu z systemu. 
		\hfill 3
	\end{myEnumerate}

	\item\label{f17} Obsługa innego sprzętu, urządzeń i wyposażenia. 
	\hfill 1
	\begin{myEnumerate}
		\item\label{f18} Dodanie zasobu. 
		\hfill 1
		\begin{myEnumerate}
			\item\label{f19} Dodanie podstawowych informacji o zasobie. 
			\hfill 1
		\end{myEnumerate}
		\item\label{f20} Edycja informacji o zasobie. 
		\hfill 2
		\item\label{f21} Usunięcie zasobu z systemu. 
		\hfill 3
	\end{myEnumerate}

	\item\label{f22} Obsługa czasopism oraz literatury. 
	\hfill 1
	\begin{myEnumerate}
		\item\label{f23} Dodanie zasobu. 
		\hfill 1
		\begin{myEnumerate}
			\item\label{f24} Dodanie podstawowych informacji o zasobie. 
			\hfill 1
		\end{myEnumerate}
		\item\label{f25} Edycja informacji o zasobie. 
		\hfill 2
		\item\label{f26} Usunięcie zasobu z systemu. 
		\hfill 3
	\end{myEnumerate}

	\end{myEnumerate}

	\item\label{f27} Współpraca z modułem rejestru pracowników. 
	\hfill 1
	\begin{myEnumerate}
	\item\label{f28} Zapisanie informacji o osobie odpowiedzialnej za dany zasób. 
	\hfill 1
	\item\label{f29} Zapisanie informacji o użytkowniku konkretnego
	 zasobu. 
	 \hfill 1
	\end{myEnumerate}

	\item\label{f30} Obliczanie statystyk. 
	\hfill 3
	\begin{myEnumerate}
	\item\label{f31} Prezentacja zakupów konkretnych zasobów w poszczególnych \\ latach. 
	\hfill 3
	\item\label{f32} Prezentacja ilości zasobów w poszczególnych działach. 
	\hfill 3
	\item\label{f33} Prezentacja ilość zasobów w poszczególnych placówkach. 
	\hfill 3
	\end{myEnumerate}
	\end{myEnumerate}


\item\label{f34} \textbf{Zarządzanie serwisem sprzętu oraz oprogramowania.}
	\hfill 1
	\begin{myEnumerate}
		\item\label{f35} Rejestracja informacji dotyczącej miejsca zakupu.
			\hfill 1
		\item\label{f36} Historia napraw sprzętu. 
		\hfill 1
			\begin{myEnumerate}
			\item\label{f37} Rejestracja informacji o wewnętrznej naprawie sprzętu. 
			\hfill 1
			\item\label{f38} Rejestracja naprawy sprzętu przez serwis zewnętrzny.  
			\hfill 1
			\end{myEnumerate}
		\item\label{f39} Historia obsługi oprogramowania. 
		\hfill 1
		\begin{myEnumerate}
		\item\label{f40} Rejestracja aktualizacji oprogramowania.
			\hfill 1
		\item\label{f41} Rejestracja "naprawy" oprogramowania.
					\hfill 1
		\end{myEnumerate}
		\item\label{f42} Współpraca z repozytorium problemów.
					\hfill 1
		\begin{myEnumerate}
		\item\label{f43} Rejestracja operacji serwisowej.
					\hfill 1
		\end{myEnumerate}
	\end{myEnumerate}
\end{myEnumerate}

% ostatnia etykieta labelki: f43

\subsection{Wymagania niefunkcjonalne}


\begin{myEnumerate}
	\item \textbf{Bezpieczeństwo}
	\begin{myEnumerate}
		\item Przechowywanie danych użytkowników zgodnie z wymogami ustawy o ochronie danych osobowych.
		\item Szyfrowanie danych wrażliwych dotyczących użytkowników zasobów.
		\item Komunikacja z systemem z wykorzystaniem SSL.
		\item Hasła użytkowników przetrzymywane w formacie bezpiecznym - w postaci wyniku funkcji skrótu hasła połączonego z losowo wygenerowaną solą.
		\item Integracja z firmowym serwerem LDAP.
	\end{myEnumerate}
	\item \textbf{Dostępność}
	\begin{myEnumerate}
		\item Dostępność systemu na poziomie 99\%.
	\end{myEnumerate}
	\item \textbf{Niezawodnosć}
	\begin{myEnumerate}
		\item Możliwość wykonywania kopii zapasowych.
		\item Możliwość zautomatyzowania procesu wykonywania kopii zapasowych.
		\item Maksymalny czas restartu systemu 2h.
		\item Możliwość odtworzenia systemu z kopii zapasowej w czasie poniżej 6h.
		\item Transakcyjny charakter operacji.
	\end{myEnumerate}
	\item \textbf{Użyteczność}
	\begin{myEnumerate}
		\item Intuicyjny interfejs graficzny.
		\item Wbudowana pomoc kontekstowa dla użytkowników systemu.
		\item Przygotowanie materiałów szkoleniowych.
		\item System dostępny w wersjach językowych: polskiej, koreańskiej.
	\end{myEnumerate}
	\item \textbf{Elastyczność}
	\begin{myEnumerate}
		\item Architektura systemu zapewnia możliwość dodawania nowych funkcji.
		\item Architektura systemu zezwala na zintegrowanie go z innymi systemami.
		\item Możliwość skalowania systemu.
	\end{myEnumerate}
	\item \textbf{Wydajność}
	\begin{myEnumerate}
		\item System przystosowany jest do równoczesnej pracy 100 pracowników.
		\item Czas średniej odpowiedzi systemu na zapytanie powinien być krótszy niż 300 ms.
	\end{myEnumerate}
\end{myEnumerate}

\subsection{Specyfikacja przypadków użycia - poziom ogólny}
\begin{longtable}{| c | c | p{.20\textwidth} | p{.40\textwidth} |} 

	\hline \textbf{ID} & \textbf{Wymaganie} & \textbf{Nazwa} & \textbf{Opis} \\ 
	\hline UC1 & 1.1. & Wyszukiwanie zasobów  & Użytkownik wyszukuje zasoby podając kryteria wyszukiwania \\ 
	\hline UC2.1 & 1.2.1.1. & Dodanie nowego sprzętu komputerowego do systemu & Użytkownik dodaje nowy sprzęt komputerowy do systemu katalogowania \\ 
	\hline UC2.2 & 1.2.2 & Edycja informacji o sprzęcie komputerowym & Użytkownik edytuje informacje dotyczące sprzętu komputerowego, który już znajduje się w systemie \\ 
	\hline UC2.3 & 1.2.3. & Usunięcie sprzętu komputerowego z systemu & Użytkownik usuwa sprzęt komputerowy z systemu \\ 
	\hline UC3.1 & 1.3.1. & Dodanie nowego oprogramowania do systemu & Użytkownik dodaje nowego oprogramowanie do systemu katalogowania \\ 
	\hline UC3.2 & 1.3.2. & Edycja informacji o oprogramowaniu & Użytkownik edytuje informacje dotyczące oprogramowania znajdującego się w systemie \\ 
	\hline UC3.3 & 1.3.3. & Usunięcie oprogramowania z systemu & Użytkownik usuwa informacje dotyczące oprogramowania z systemu  \\ 
	\hline UC4.1 & 1.4.1. & Dodanie innego sprzętu, urządzeń lub wyposażenia do systemu & Użytkownik dodaje nowy sprzęt, urządzenie lub wyposażenie do systemu \\ 
	\hline UC4.2 & 1.4.2. & Edycja informacji o innym sprzęcie, urzędzeniu lub wyposażeniu & Użytkownik edytuje informacje dotyczące innego sprzętu, urządzenia lub wyposażenia znajdującego się w systemie \\ 
	\hline UC4.3 & 1.4.3. & Usunięcie innego sprzętu, urządzenia lub wyposażenia z systemu & Użytkownik usuwa informacje dotyczące innego sprzętu, urządzenia lub wyposażenia z systemu \\
	\pagebreak
	UC5.1 & 1.5.1. & Dodanie nowego czasopisma lub zasobu literaturowego do systemu & Użytkownik dodaje nowego czasopismo lub zasób literaturowy do systemu katalogowania \\ 
	\hline UC5.2 & 1.5.2. & Edycja informacji o czasopiśmie lub zasobie literaturowym & Użytkownik edytuje informacje dotyczące czasopisma lub zasobu literaturowego znajdującego się w systemie \\ 
	\hline UC5.3 & 1.5.3.  & Usunięcie czasopisma lub zasobu literaturowego z systemu & Użytkownik usuwa informacje dotyczące czasopisma lub zasobu literaturowego z systemu \\ 
	\hline UC6.1 & 1.6.1. & Zapisanie informacji o osobie odpowiedzialnej za zasób & Użytkownik zapisuje informację o osobie odpowiedzialnej za konkretny zasób \\ 
	\hline UC6.2 & 1.6.2. & Zapisanie informacji o użytkowniku konkretnego zasobu & Użytkownik zapisuje informację o użytkowniku konkretnego zasobu \\ 
	\hline UC7.1 & 1.7.1. & Prezentacja zakupów konkretnych zasobów w poszczególnych latach & Użytkownik ma możliwość zobaczenia statystyk dot. zakupów konkretnych zasobów w poszczególnych latach \\ 
	\hline UC7.2 & 1.7.2. & Prezentacja ilości zasobów w poszczególnych działach & Użytkownik ma możliwość zobaczenia statystyk dot. zasobów w poszczególnych działach \\ 
	\hline UC7.3 & 1.7.3. & Prezentacja ilości zasobów w poszczególnych placówkach & Użytkownik ma możliwość zobaczenia statystyk dot. zakupów konkretnych zasobów w poszczególnych placówkach \\ 
	\hline UC8 & 2.1. & Rejestracja informacji o miejscu zakupu zasobu & Użytkownik rejestruje miejsce zakupu konkretnego zasobu \\ 
	\hline UC9.1 & 2.2.1. & Rejestracja informacji o wewnętrznej naprawie sprzętu & Użytkownik rejestruje fakt przeprowadzenia wewnętrznej naprawy sprzętu \\ 
	\hline UC9.2 & 2.2.2. & Rejestracja informacji o zewnętrznej naprawie sprzętu & Użytkownik rejestruje fakt przeprowadzenia zewnętrznej naprawy sprzętu \\ 
	\hline UC10 & 2.3.1. & Rejestracja aktualizacji oprogramowania & Użytkownik rejestruje przeprowadzenie aktualizacji oprogramowania \\ 
	\hline UC11 & 2.4.1. & Rejestracja operacji serwisowej & Użytkownik rejestruje przeprowadzenie operacji serwisowej zasobu \\ 
	\hline 
\end{longtable} 

